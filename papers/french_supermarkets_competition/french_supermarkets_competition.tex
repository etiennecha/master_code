\documentclass[english]{article}
\usepackage[T1]{fontenc}
\usepackage[latin1]{inputenc}
\usepackage{enumerate}
\usepackage{setspace}
\usepackage{amsmath,amssymb,amsthm}
\usepackage{graphicx}
\usepackage{bbm}
\usepackage[round]{natbib}
\usepackage[nohead]{geometry}
\usepackage[bottom]{footmisc}
\usepackage{indentfirst}
\usepackage{endnotes}
\usepackage{graphicx}%
\usepackage{eurosym}
\usepackage{array}
\usepackage{booktabs}
\usepackage{caption}
\usepackage{subcaption}
\usepackage{tabularx}
\usepackage[flushleft]{threeparttable}
% \usepackage[hidelinks]{hyperref}
\usepackage{floatrow} %[capposition=top]
\floatsetup{footposition=bottom,capposition=top}
\renewcommand{\labelitemi}{--}
\renewcommand{\labelitemii}{$\bullet$}
\bibliographystyle{chicago}
% \geometry{left=1in,right=1in,top=1.00in,bottom=1.0in}
\let\olditemize\itemize
\renewcommand{\itemize}{
  \olditemize
  \setlength{\itemsep}{-1pt}
}

\begin{document}

\title{Competition between French grocery stores: Evidence from a price comparison website\ \\ \ \\Working paper}
\author{E. Chamayou\thanks{e-mail:
\textit{etienne.chamayou@ensae.fr}} \\ CREST-LEI}
\maketitle

\sloppy%

\onehalfspacing

\textbf{Abstract:}

The French grocery store chain Leclerc operates a price comparison website which allows to compare each of its stores with some local competitors, and performs chain comparisons at the national level. This papers uses price data collected from the comparison website to investigate static and dynamic price dispersion across French grocery stores. Although chains are found to largely determine store pricing policies, chain level comparisons are often of little informative value given heterogeneity observed at the store level. Furthermore, store comparison results tend to be sensitive to available product sets and vary significantly over time. Findings therefore suggest that static and dynamic price dispersion makes accurate price comparisons very costly for consumers.

\strut

\textbf{Keywords:}

\strut

\textbf{JEL Classification Numbers:} D83, L81, M31

\pagebreak%
%\doublespacing

\section{Introduction}

Since the development of supermarket chains in France, several regulations have been implemented with a view to protect smaller retailers and producers from increasingly large retail chains. The academic literature has yet painted an unflattering portrait of passed laws.

%such as the Loi Raffarin, which tightened restrictions on new store openings in 1996, or the Loi Galland, aimed at effectively forbiding below cost pricing in 1997.

\cite{BER02} analyse the impact of a restriction on large store openings introduced in 1974 to protect small retail stores. They find that a stronger deterrence of entry, decided by boards at the regional level, is associated with increased retailer concentration and weaker employment growth.

\cite{BIS13} study the effects of the Loi Galland, passed in 1997, which modified existing below-cost pricing regulations with a view to protect small retailers and producers from larger retail chains. Existing regulation, dating back to 1963, had indeed proved to be inefficient given its loose definition of cost. The new law was thus meant to clarify the rules by defining the threshold as the invoice price. This forbade to take ex-post rebates into account in the final price. The paper documents a weakening of the relation between concentration and retail prices which is likely to reflect a reduction in intra-brand competition resulting from the Loi Galland. Indeed, the new regulation gives suppliers of branded products the possibility to impose industry-wide price floors (minimum RPM), while negotiating the actual wholesale price with retailers through rebates. As expected, the measured effect is stronger for branded product than for store brand products, which are less likely to have been by the change in the regulation. Price dispersion is found to be reduced for branded products and the price gap between most expensive and less expensive stores is reduced. Price dispersion is yet observed to remain significant.

Since the seminal paper of \cite{STI61}, a large literature has investigated the link between "consumer ignorance" and price dispersion i.e. the persistence over time of different prices for a homogeneous good in a given market. Following \cite{VAR80}, a rich theoretical paradigm has emerged in which price dispersion results from price randomization by sellers in equilibrium. Empirical research, on the other hand, has long been hampered by a scarcity of relevant panel data.

\cite{LAC02} studies price dispersion with CPI data of four grocery store products sold in Israel over four years. Dynamic price dispersion is documented in the form of sellers frequently changing quartiles in the price distribution over months at the national level. Data are yet not rich enough to observe price dispersion within local markets.

\cite{ZHA06} investigates the relation between price dispersion, measured through the coefficient of variation, with consumer search costs, competition intensity, and consumer heterogeneity. A positive correlation with dispersion is found for each of these elements. Data consist in a scanner panel covering 23 product categories of 6 supermarkets within a surburban area of Chicago from June 1991 to Junz 1993. The analysis focuses on the largest 10 brands in terms of market share within each product category. 

Product categories are narrow enough (e.g. Butter, Coffee) for products within categories to be considered as alternatives for a given consumer. Price dispersion related to product size and brand diversity can then be quantified. The unit price of a given product (as defined by brand and quality) is found to generally decrease with size, while significant heterogeneity across intrabrand price dispersion is measured, meaning that brands exhibit different levels of differentiation.

The paper uses the frequency of store visits and the frequency of product category or brand purchases as proxies for search costs. Intensity of competition varies due to a store entry in the market over the studied period. Higher industry concentration is expected to lead to lower price dispersion. Consumer heterogeneity is measured through the coefficients of variation in various consumer demographic variables. Lower variability is expected to involve a lower role of price discrimination hence lower dispersion.

\cite{TAP11}, with retail gasoline price data, analyse price spreads of pairs of competitors based on the distance which separate them. They remark that the price ranking between two competitors tends to be noisier when distance increases, namely when consumer search costs are more likely to relax competition and therefore to create the tension between rent extraction and undercutting that leads to a mixed strategy equilibrium in \cite{VAR80}.

\cite{PER15} analyse price dispersion in the French supermarket industry with four product categories (beer, cola, coffe and whisky) with data spanning 1999-2001. They find that stores frequently change positions in the price ranking and estimate a structural model which accomodates sequential search, vertical product differentiation and heterogeneous consumer tastes. They find that search cost are high and that the majority of consumers is thus poorly informed about prices in equilibrium. Price elasticities differ significantly from the perfect information model.

\cite{TUR16} estimates a structural model of demand and recover stores' price-cost margins. The paper finds that stores set prices according to the most competitive scenario, but that a significant portion of large grocery stores take advantage of insufficient competitive pressure to distort offer and increase margins.

In a theoretical paper, \cite{ALL16a} investigate the consequences of spatial discrimination and uniform pricing strategies on mergers. They show that when one retailer implements uniform pricing, the anticompetitive effects of a merger can affect consumers in markets not directly affect by the merger.

Using rich price data collected from a comparison website, this paper documents static and dynamic price dispersion in the French supermarket industry. The first part of the paper provides a description of the data and reviews the methodology used by the comparison website from which they are extracted. The second part discusses the relevance of chain comparisons and investigate price dispersion at the store and market level. Chain affiliation is found to be a strong determinant of prices, while market characteristics account for a small share of price variations. Leclerc and Geant Casino, the cheapest chains as of March 2015, exhibit highest degrees of price uniformity compared to other chains. Whenever the average price difference across products between two competing stores is small enough, the result of the price comparison largely depends on product choice. TODO: result description.

\section{Price comparison on quiestlemoinscher.com}

As of 2015, the French retailing industry was dominated by six firms, which accounted for over 80\% of total sales: Carrefour and Leclerc were the two largest with c. 20\% market shares each, followed by Intermarche (14\%), Casino (12\%), Auchan (11\%) and Systeme U (10\%). An important distinction between firms lies in the ownership structures. While Carrefour, Casino and Auchan own most of the stores operated under their brands, Leclerc, Intermarche and Systeme U are essentially franchise networks.

The creation of the comparison website quiestlemoinscher (thereafter "Qlmc") is part of a long term strategy of the group Leclerc to prove the competitiveness of its prices. Soon after the launch in May 2006, Carrefour filed a complaint about the lack of transparency and potential biases in comparisons. The website was forced to close by a court decision. An updated version of the website was released on November 2006 and has since then remained in operation. Legal proceedings nervetheless continued until the rejection by the court of cassation of Carrefour's claims in January 2010. A major merit of the legal action undertaken by Carrefour is its consequence for the transparency, namely the release of well identified store product price data. The following section provides an overview of the methodology of the comparison website, two crucial aspects of which are competitor and product choice. Data collected from the website are use to replicate its comparison results. The website was explored in March 2015, with a view to find the best way to extract price data. The only feasible solution appeared to successively crawl comparisons between Leclerc stores and each of their competitors. This implies that obtained data differ from the price database used by Qlmc to establish comparisons performed at the national level with competing supermarket chains.

\subsection{Competitors}

Until 2013, the website only offered comparisons between Leclerc and its competitors at the chain level. For each competing chain, prices were collected at a sample of store expected to be representative of the store network. Broad constraints were thus imposed on store location and size, while exact store choice was claimed to be random. From 2013 on, the development of the "drive" concept in France has allowed the comparison website to cover far more stores, and thus to start displaying store level comparisons. The concept of "drive" implies that consumers are offered the opportunity to shop online from a physical store (at the same prices) and collect their purchases whenever it suits them. The collection of prices can then be achieved efficiently on the internet, as opposed to costly physical store visits. As of March 2015, Qlmc claimed to cover 60\% of the stores of the 10 supermarket chains compared (44\% in August 2013).

Regarding store level comparisons, the website states that each Leclerc is compared with a selection of its most relevant competitors within 30 km, based on Leclerc managers' expertise. The website also indicates that stores whose surface is smaller than 1,000 $m^2$ are excluded, as well as stores belonging to chains which are deemed to be too differentiated (e.g. discounters). Finally, Leclerc stores are not included among potential competitors. A total number of 575 Leclerc stores were found to be listed on the website in March 2015. The comparison of each store with its respective selection of competitors yielded 2,390 pairs of stores, involving 1,815 unique Leclerc competitors (the number of pairs is larger since a store can be listed as a competitor of several Leclerc stores). Data were missing for 14 Leclerc stores and 51 Leclerc competitors. This implies that among competitors of the 561 Leclerc stores for which price data have been collected, 36 out of 1811 are missing ($\le 2 \%$).

\begin{table}[H]
\begin{threeparttable}
\renewcommand{\arraystretch}{0.7}% Tighter
\caption{Representation of major national chains}
\label{tab:qlmc_chain_repr}
\small
\begin{tabular}{lr|rr|rr}
\toprule
          & France & \multicolumn{2}{c|}{QLMC} & \multicolumn{2}{c}{Data} \\
          & Nb stores & Nb stores & Coverage & Nb stores & Coverage \\
\midrule
    Auchan & 142   & 125   & 88\%  & 112   & 79\% \\
    Carrefour & 222   & 188   & 85\%  & 171   & 77\% \\
    Carrefour Market & 925   & 421   & 46\%  & 239   & 26\% \\
    Casino & 392   & 151   & 39\%  & 76    & 19\% \\
    Cora  & 58    & 58    & 100\% & 54    & 93\% \\
    Geant & 108   & 108   & 100\% & 92    & 85\% \\
    Intermarche & 1,770 & 1,022 & 58\%  & 530   & 30\% \\
    Leclerc & 579   & 579   & 100\% & 561   & 97\% \\
    Simply Market & 305   & 50    & 16\%  & 49    & 16\% \\
    Systeme U & 1,030 & 632   & 61\%  & 413   & 40\% \\
\midrule		
    Total & 5,531 & 3,334 & 60\%  & 2,297 & 42\% \\
\bottomrule
\bottomrule
\end{tabular}
\begin{tablenotes}
      \small
      \item Data about store chains were provided by LSA, which is the source used by Qlmc.
\end{tablenotes}
\end{threeparttable}
\end{table}

Table~\ref{tab:qlmc_chain_repr} provides an overview of store coverage for the ten national chains compared on Qlmc in March 2015. Coverage is high and rather close to coverage in the full Qlmc sample for chains which are characterized by large store surfaces: Auchan, Carrefour, Cora, Geant and Leclerc. This can be explained by the fact that Leclerc is present across all regions and operates rather large stores. Regarding chains with smaller store formats, coverage is lower both for Qlmc and in the data with respect to the website (e.g. 22\% for Casino vs. 39\% claimed by the website). Two natural explanations are the fact that stores from these chains are generally less likely to be considered relevant local competitors for Leclerc stores, and the slower development of "drive" within smaller stores (which make price collection less costly).

\begin{table}[H]
\renewcommand{\arraystretch}{0.7}% Tighter
\caption{Overview of competition around the 575 Leclerc stores in Qlmc}\label{tab:qlmc_comp}
\small
\begin{tabular}{lr|rrrr}
\toprule
\toprule
{}         & Nb          &    \multicolumn{4}{c}{Distance (km) to} \\
{}         & competitors &    closest & furthest & mean & median \\
\midrule
Mean  &   5.0 &   2.4 &  15.9 &   8.8 &     8.5 \\
Std   &   1.6 &   2.5 &   9.7 &   5.1 &     6.0 \\
Min   &   1.0 &   0.1 &   0.9 &   0.8 &     0.5 \\
Q10   &   3.0 &   0.7 &   4.6 &   3.0 &     2.5 \\
Q25   &   4.0 &   1.1 &   8.4 &   4.8 &     3.7 \\
Q50   &   5.0 &   1.8 &  15.3 &   7.8 &     6.5 \\
Q75   &   6.0 &   2.7 &  21.5 &  12.3 &    12.5 \\
Q90   &   7.0 &   4.7 &  26.3 &  15.7 &    18.0 \\
Max   &  12.0 &  21.1 &  67.0 &  28.6 &    28.5 \\
\bottomrule
\bottomrule
\end{tabular}
\end{table}

Table~\ref{tab:qlmc_comp} provides an overview of competition according to qlmc comparisons (the website does not claim to be comprehensive). On average, a Leclerc store is compared with 5 competitors, and over 50\% over Leclerc stores are compared with a store located within 2 km. The furthest competitor is generally within 30 km (15 km for almost half of them), except for 28 stores. For 14 Leclerc stores, the closest listed store if over 10 km away. No store meets these two criteria, hence it does not seem obvious that the lack or omission of nearby competitors led to include stores beyond reasonable distance. For instance, the Leclerc outlet which has the furthest competitor in the data (67 km) is listed with 7 competitors, of which 5 are located within 30 km.

\subsection{Products and methodology}

As of March 2015, only national brand products are covered by the website. Products are identified at the bar code level. There are seven food product categories: meat and fish, vegetables and fruits, bakery, fresh food, frozen food, savoury grocery, sweet grocery, baby food and drinks. Non food products are split in four categories: health and beauty, household, pets and home and textile. Products are further classified within product families. Regarding chain comparisons, the number of products covered within each family is determined by the volume of national hypermarket and supermarket sales, with a global objective of 3,000 products. Within each family, products are chosen based on the national hypermarket and supermarket detention rate. Products whose detention rate is below 30\% (i.e. products referenced by less than 30\% of the stores) are dropped. This led to a total of 2,461 national brand product references covered for March 2015 (2,510 in August 2013). As regards store level comparisons, all products found at both stores are used in comparisons.

Price records obtained from the website include all products used in each store level comparison. As a consequence, there are 12,318 unique products in the data as of March 2015. Table~\ref{tab:qlmc_sections} provides an overview of the relative weights of each section in terms of product number and value. The five largest sections, regardless of the criterion, are Fresh products, Health and Beauty, Savoury Grocery, Sweet Grocery and Drinks. Families within each sections are detailed in table~\ref{tab:qlmc_families}. Drinks and Health and Beauty products tend to have larger values than products from other categories, so that they account for a significantly higher share in terms of value than product count.

\begin{table}[H]
\renewcommand{\arraystretch}{0.7}% Tighter
\caption{Overview of product section weights}\label{tab:qlmc_sections}
\small
\begin{tabular}{lrr|rr|rr}
\toprule
\toprule
& \multicolumn{2}{c}{All products} & \multicolumn{2}{c|}{$\ge$ 500 obs} & \multicolumn{2}{c|}{$\ge$ 700 obs} \\
& Nb \% & Value \% & Nb \% & Value \% & Nb \% & Value \% \\
\midrule
    Baby and dietetic food & 4.7   & 4.3   & 3.9   & 3.0   & 3.3   & 2.4 \\
    Drinks & 10.0  & 15.3  & 10.9  & 20.4  & 11.1  & 21.9 \\
    Fresh products & 21.1  & 15.5  & 19.8  & 16.7  & 18.4  & 15.2 \\
    Frozen food & 3.0   & 3.1   & 3.0   & 3.9   & 2.4   & 3.1 \\
    Health and beauty & 17.3  & 26.9  & 11.5  & 12.8  & 12.4  & 13.4 \\
    Home and textile & 2.5   & 3.4   & 0.5   & 0.7   & 0.3   & 0.4 \\
    Household & 5.5   & 6.8   & 5.5   & 6.8   & 5.8   & 7.2 \\
    Pets  & 1.9   & 2.8   & 3.0   & 4.4   & 3.0   & 4.5 \\
    Savoury grocery & 16.5  & 9.4   & 19.6  & 12.5  & 20.4  & 12.6 \\
    Sweet grocery & 17.0  & 12.3  & 22.1  & 18.8  & 22.8  & 19.2 \\
    Vegetables and fruits & 0.5   & 0.4   & 0.2   & 0.2   & 0.2   & 0.2 \\
\midrule
    Total & 100.0 & 100.0 & 100.0 & 100.0 & 100.0 & 100.0 \\
    Total Nb or Value (euros) & 12,318 & 43,883 & 3,467 & 9,138 & 2,578 & 6,682 \\
\bottomrule
\bottomrule
\end{tabular}
\end{table}

The comparison of Leclerc with its competitors follows two simple steps. First, the average price of each product is computed for each chain, provided the product is observed within enough stores of the chain. Leclerc is then successively compared to each of its competitors based on all products for which a chain price was computed. The result displayed on the website is the percentage difference between the price of the basket for the competing chain and for Leclerc:
\begin{align*}
\frac{\sum\limits_{i} P_{iC} - \sum\limits_{i} P_{iL}}{\sum\limits_{i} P_{iL}}
\end{align*}
where $i$ refers to all products in the baskets, $P_{iC}$ and $P_{iL}$ respectively stand for the average price of product $i$ for the competing chain ($C$) and for Leclerc ($L$). The comparison between two stores is very similar except that it uses store prices instead of average chain prices.

\subsection{Price comparison results}

Results for chain level comparisons performed according to the website methodology are reported in table~\ref{tab:qlmc_chain_comparisons}. Despite the fact that data collected differ from these used by Qlmc, results are very similar, and are found to be relatively robust to variations in included products. Geant Casino is the second cheapest chain as of March 2015, only 1.5\% more expensive than Leclerc (1.8\% according to Qlmc). Dropping the 20\% products which weigh in most favorably for Leclerc reduces the comparison result to 0.4\%.

\begin{table}[H]
\caption{Comparisons at the chain level}
\label{tab:qlmc_chain_comparisons}
\begin{threeparttable}
\renewcommand{\arraystretch}{0.7}% Tighter
\small
\begin{tabular}{l|rr|rr|rrrr}
\toprule
\toprule
          & \multicolumn{2}{c|}{Nb stores} &  \multicolumn{2}{c|}{Nb products} & \multicolumn{4}{c}{Comparison vs. Leclerc} \\
           & Qlmc  & Data  & Qlmc  & Data  & Qlmc  & Data  & Bias 10\% & Bias 20\% \\
\midrule
    Auchan & 125   & 112   & 1,976 & 2,382 & +7.6\% & +6.5\% & +5.5\% & +5.0\% \\
    Carrefour & 188   & 171   & 1,294 & 1,284 & +7.8\% & +8.2\% & +7.0\% & +6.0\% \\
    Carrefour market & 421   & 239   & 2,032 & 3,401 & +13.5\% & +12.4\% & +11.6\% & +10.2\% \\
    Casino & 151   & 76    & na    & 1,650 & +16.7\% & +16.8\% & +15.8\% & +15.4\% \\
    Cora  & 58    & 54    & 1,326 & 2,994 & +10.2\% & +9.4\% & +8.3\% & +7.3\% \\
    Geant Casino & 108   & 92    & 1,582 & 1,582 & +1.8\% & +1.5\% & +0.7\% & +0.4\% \\
    Intermarche & 1,022 & 530   & 1,971 & 6,287 & +7.0\% & +7.1\% & +5.8\% & +5.0\% \\
    Simply market & 50    & 49    & na    & 1,070 & +12.9\% & +13.4\% & +11.6\% & +11.2\% \\
    Systeme U & 632   & 413   & 2,386 & 4,565 & +6.7\% & +5.8\% & +4.8\% & +4.7\% \\
\bottomrule
\bottomrule
\end{tabular}
\begin{tablenotes}
      \small
      \item Comparisons are based on 561 Leclerc stores (vs. 581 in Qlmc). In the column "Bias 10\%", the 10\% products which compare most favorably for Leclerc in terms of percent price difference are dropped.
\end{tablenotes}
\end{threeparttable}
\end{table}

Results for store level comparisons performed according to the website metholodgy are reported in table~\ref{tab:qlmc_store_comparisons}. The website lists 99 comparisons between a Leclerc and a Geant Casino. On average, the Geant Casino is found to be 1.8\% more expensive than its Leclerc competitor.  In some cases, Geant Casino is cheaper than Leclerc. A Geant Casino is indeed found to be 0.6\% less expensive than its Leclerc competitor. In general, national level comparisons appear to be relatively representative considering the distributions of store level comparisons.

\begin{table}[H]
\caption{Comparisons between Leclerc stores and their competitors by chain}
\label{tab:qlmc_store_comparisons}
\begin{threeparttable}
\renewcommand{\arraystretch}{0.7}% Tighter
\small
\begin{tabular}{lr|rrrrrrr}
\toprule
\toprule
          & Nb    & \multicolumn{7}{c}{Comparison of Leclerc stores vs. competitors by chain} \\
          & pairs & Mean  & Std   & Min   & Q25  & Q50 & Q75  & Max \\
\midrule
    Auchan & 118   & +6.5\% & 3.3\% & +1.6\% & +4.1\% & +5.7\% & +8.3\% & +19.5\% \\
    Carrefour & 175   & +8.2\% & 5.2\% & -3.5\% & +5.8\% & +8.1\% & +9.4\% & +36.2\% \\
    Carrefour market & 235   & +13.8\% & 3.3\% & +1.3\% & +11.7\% & +13.5\% & +15.8\% & +24.5\% \\
    Casino & 57    & +17.9\% & 4.8\% & +0.5\% & +16.8\% & +18.7\% & +21.0\% & +27.5\% \\
    Cora  & 57    & +8.6\% & 2.4\% & +3.6\% & +6.7\% & +8.4\% & +10.3\% & +15.6\% \\
    Geant Casino & 99    & +1.8\% & 1.5\% & -0.6\% & +0.7\% & +1.3\% & +2.3\% & +5.3\% \\
    Intermarche & 525   & +7.1\% & 2.8\% & +2.0\% & +5.4\% & +6.6\% & +8.2\% & +28.4\% \\
    Simply market & 49    & +13.4\% & 6.2\% & +6.5\% & +9.8\% & +10.6\% & +15.4\% & +31.8\% \\
    Systeme U & 355   & +6.7\% & 4.0\% & +1.1\% & +3.8\% & +5.8\% & +8.7\% & +26.0\% \\
\bottomrule
\bottomrule
\end{tabular}
\begin{tablenotes}
      \small
      \item Pairs are required to include 400 products or more. There are 118 comparisons between a Leclerc store and an Auchan store. On average, an Auchan store is 6.5\% more expensive than its Leclerc competitor.
\end{tablenotes}
\end{threeparttable}
\end{table}

\subsection{Comparison history}

Price records can be used to compute variations in chain prices across periods, which allows to gain some understanding regarding the evolution in price comparison results observed over time. The methodology is the very same as the one used to compare two chains in a given period, except that the comparison involves all the product prices of one chain for which an average price can be computed in the two periods of interest. Variations can then be chained to obtain statistics over longer periods. Indeed, product turnover generally does not allow meaningful direct comparisons between non successive price records.

Leclerc prices between May 2007 and May 2012 have increased by 1.13\% (average annual increase of 0.25\%). Until May 2011, other chain display similarly low variations. This translates in a relative status quo in chain comparison results. Geant Casino is then the most expensive chain relative to Leclerc (from +6\% to +10\%), followed by Cora and Carrefour Market (+5\%). Auchan, Carrefour, Geant Casino, Intermarche and Systeme U display rather similar price levels (+3\% to +4\%). After May 2011, most chains exhibit a progressive loss in competitivess as compared to Leclerc. Geant Casino, however, constitutes a remarkable exception. After a peak in September 2012 (13.8\% more expensive than Leclerc), the chain becomes increasingly price competitive from May 2013 on. As of March 2015, Geant Casino is the closest competitor of Leclerc in terms of price level (+1.3\% vs. Leclerc), while it was actually the most expensive chain at the beginning of the period, and was still 12.2\% more expensive than Leclerc as of March 2013. The history of comparisons also reveals that Carrefour, after a progressive increase in price competitiveness in the second half of 2013 and the first half of 2014 (+2.6\% vs. Leclerc in September 2014), catches up abruptly with other comparable chains (Auchan, Intermarche and Systeme U) in March 2015 which are between 6\% and 7\% more expensive than Leclerc.

Intra-chains comparisons between May 2014 and March 2015 suggest that the relative loss of price competitiveness exhibited by Carrefour actually results from a mild change in prices by Carrefour (-1.4\%) constrasting with significant cuts implemented by other chains (e.g. -4.3\% for Auchan, -5.1\% for Leclerc, -5.2\% for Intermarche). Geant Casino achieves its unprecedented level of price competitiveness through an 8.5\% decrease.

Overall, the history of comparisons reveals that beyond some stability at both extremities of the price ranking (Cora and Carrefour Market are persistently found to be relatively expensive while Leclerc is always the cheapest chain), one chain, Geant Casino, radically changes its pricing policy in less than a year, and the ranking between the remaining national chains (Auchan, Carrefour, Intermarche and Systeme U) exhibits significant volatility over time.

\section{Price dispersion}

Since its creation in 2007, Qlmc prominently displays aggregate comparisons with its major national competitors. On the one hand, such information may be considered relevant by consumers willing to adopt a rule of thumb which weighs the cost/time of transportation with prices and products expected to be offered in a store of a given chain. On the other hand, without being deceptive per se, such comparisons could simply reflect differences in store characteristics such as location, size, market competitiveness etc. In order to address this issue, this section first investigates price dispersion within chains, and then discusses the possibility to account for prices through observed store and market characteristics.

\subsection{Chain level pricing policies}

French supermarket chains are known not to generally follow uniform national pricing policies. Empirical investigations however reveal various degrees of uniformity at the chain level. Table~\ref{tab:qlmc_prod_freq} reports the frequency of the mode (most common price) of each product at the chain level. Geant Casino stands out in terms of product price homogeneity. On average, a product is sold at the very same price in 89\% of the chain stores. This implies that a random basket of goods has a relatively high probability to have the very same price in two Geant Casino stores, even if both are located far apart from each each other. The closest followers are Systeme U and Leclerc, for which the mode accounts for 39\% and 38\% of price observations on average.

\begin{table}[H]
\caption{Distribution of the frequency of the mode (most common price) per product}
\label{tab:qlmc_prod_freq}
\begin{threeparttable}
\renewcommand{\arraystretch}{0.7}% Tighter
\small
\begin{tabular}{lrrrrrrrr}
\toprule
\toprule
{}                 &  Nb &  Mean &  Std &  Min &  Q25 &  Q50 &  Q75 &  Max \\
\midrule
Auchan & 416   & 19    & 11    & 5     & 12    & 16    & 22    & 63 \\
Carrefour & 319   & 29    & 17    & 7     & 17    & 23    & 36    & 87 \\
Carrefour Market & 777   & 33    & 19    & 11    & 20    & 26    & 42    & 100 \\
Geant Casino & 417   & 89    & 10    & 45    & 83    & 91    & 97    & 100 \\
Casino & 157   & 37    & 15    & 6     & 29    & 33    & 44    & 86 \\
Cora  & 364   & 20    & 11    & 6     & 14    & 17    & 23    & 90 \\
Intermarche & 1,326 & 25    & 19    & 5     & 13    & 18    & 29    & 97 \\
Leclerc & 1,788 & 38    & 23    & 3     & 14    & 38    & 59    & 95 \\
Super U & 1,077 & 39    & 12    & 9     & 32    & 37    & 44    & 91 \\
\bottomrule
\bottomrule
%\multicolumn{9}{p{0.8\textwidth}}{\footnotesize Reading note: On average, 38\% of all Leclerc stores set the very same price for a given product.}
\end{tabular}
\begin{tablenotes}
      \small
      \item On average, 38\% of all Leclerc stores set the very same price for a given product.
\end{tablenotes}
\end{threeparttable}
\end{table}

Intra-brand price heterogeneity can also investigated from a store prospect. Table~\ref{tab:qlmc_store_freq} accounts for the percentage of products carried by each store the price of which is found to be equal to the mode of the observed chain prices. The average Geant Casino store appears to follow a standard chain price for approximately 80\% of its products. The median is 94\% while is the min is 6\% hence it appears that a limiter number of stores depart significantly from standard prices while price uniformity is the rule for the bulk of the store chains. Leclerc also exhibits a relatively strong concentration at the store level.

\begin{table}[H]
\caption{Distribution of the frequencies of "standard" prices per store}
\label{tab:qlmc_store_freq}
\begin{threeparttable}
\renewcommand{\arraystretch}{0.7}% Tighter
\small
\begin{tabular}{lrrrrrrrr}
\toprule
\toprule
{}                &  Nb &  Mean &  Std &  Min &  Q25 &  Q50 &  Q75 &  Max \\
\midrule
Auchan & 107   & 14    & 7     & 2     & 9     & 13    & 18    & 37 \\
Carrefour & 146   & 28    & 15    & 0     & 19    & 28    & 36    & 67 \\
Carrefour Market & 223   & 32    & 16    & 0     & 19    & 32    & 45    & 60 \\
Geant Casino & 91    & 81    & 23    & 6     & 71    & 94    & 96    & 98 \\
Casino & 74    & 16    & 11    & 2     & 7     & 13    & 27    & 49 \\
Cora  & 54    & 13    & 8     & 1     & 6     & 14    & 18    & 29 \\
Intermarche & 513   & 24    & 11    & 0     & 15    & 24    & 32    & 50 \\
Leclerc & 552   & 44    & 18    & 4     & 31    & 47    & 58    & 80 \\
Super U & 409   & 35    & 37    & 0     & 6     & 11    & 83    & 98 \\
\bottomrule
\bottomrule
\end{tabular}
\begin{tablenotes}
      \small
      \item On average, the prices of 44\% of the products carried by a Leclerc store are equal to the most common prices observed at Leclerc stores.
\end{tablenotes}
\end{threeparttable}
\end{table}

From a methodological point a view, it must be noted that the maximum values observed at the store level must be interpreted with caution. Absent standard national product prices, product price modes typically result from a few stores setting the same prices. The analysis can be refined by discarding price modes which are not followed by a large enough proportion of all chain stores. Robustness checks performed with thresholds of 33\% and 50\% confirm that Geant Casino and Leclerc stand out in terms of price concentration.

This analysis has been performed for each period of available price records. Results are similar across periods except for Geant Casino. In June 2012, the last observed period preceding its sharp increase in price competitiveness, the average product price mode accounts for 32\% of observations. This is to be compared with 82\% in May 2014. The increase in price competitiveness has thus been accompanied by a large price uniformization. Such a shock, having apparently affected a large number of markets across France in an essentially undifferentiated way, opens interesting research prospects. With quantity data, it would allow an approach similar to \cite{ALL16b} which combines a standard econometric analysis (differences in differences) with a structural approach, contributing to address the criticisms levelled by \cite{ANG10} against the empirical Industrial Organization literature\footnote{\cite{ANG10} criticize the overwhelming use of structural approaches as they generally require strong hypotheses. They call for more evidence relying on "simple, transparent empirical methods that trace a shorter route from facts to findings".}.

\subsection{Store price levels and dispersion}

In order to study the relation between price level, price dispersion, and market characteristics, we need to build measures of store price levels and store price dispersion. This is achieved by running the following regression:
\begin{align*}
\text{log }P_{ij} = \text{Product}_i + \text{Store}_j + \epsilon_{ij}
\end{align*}
where $\text{Product}_i$ is a product effect and $\text{Store}_j$ is a store effect. The coefficient $\text{Product}_i$ associated with a dummy variable equal to $1$ for products carried by store $i$ provides information regarding the price level of store $i$ compared to a reference store. The residual variation $\epsilon_{ij}$ can be interpreted as the percentage deviation of a store product price from its expected geometric mean. The average of the residuals is approximately null for each product and each store. Price dispersion is approximated at the store level by computing the standard deviation of the residuals. As a robustness check, the regression is also run successively for each supermarket chain so that the estimates of product fixed effects are chain specific. This specification is supported by table~\ref{tab:qlmc_prod_freq} and table~\ref{tab:qlmc_store_freq} as they reveal significant degrees of price homogeneity within chains. These can be seen to be relatively consistent with price dispersion measured through the standard deviation of price residuals.

Local market characteristics are found to account for a minor share of the variance in store indexes. In particular, Leclerc does not appear to be significantly less price competitive relative to competitors once one tries to capture the competitiveness of the local market, the size and revenue of the population and the surface of the store. The affiliation of a store appears to be a strong determinant of its overall price level, which is consistent with the relative strong intra-chain price concentration previously obtained and other studies on retail chain prices. \cite{HOS08} and \cite{CHA16} observe similar results with gas station prices respectively in the US and in France (even though gas station chains do not follow uniform pricing policies).

ADD REGRESSION RESULTS

Price dispersion measured at the store level is found to be strongly correlated with the store price level.

\subsection{Pair price dispersion}

Price dispersion is investigated at the pair level, based on the idea introduced in \cite{TAP11} that distance between sellers can be used as a proxy for consumer information. Pairs of competitors which are separated by a very low distance can indeed be expected to compete fiercely, so that they constitute a population in which the "law of one price" is the most likely to hold. On the other hand, a larger distance is likely to be tantamount to poorer consumer information, namely a competition setting which could be described by a model of search. A common feature of such models is the absence of pure strategy equilibria. In the single product case, mixed strategy equilibria have traditionally been given a dynamic interpretation, corresponding to the changes in ranks that can be observed among sellers over time. In the multi-product case, ADD REF have shown that sellers can randomize on each product (either in a way that simply replicates the single product case, or in a way that involves a correlation between a seller's prices).

Of all comparisons between chains exhibiting similar price levels, the Leclerc vs. Geant Casino comparison is the most stable across stores, and within stores across products. Table~\ref{tab:static_compa_15km} shows that among 215 pairs, Geant Casino is +1.4\% more expensive on average, and Leclerc is less expensive in 85.1\% of the pairs.

\begin{table}[H]
\begin{threeparttable}
\renewcommand{\arraystretch}{0.7}% Tighter
\caption{Static store level comparisons (15 km - 100 obs min)
}\label{tab:static_compa_15km}
\small
\begin{tabular}{llrrrrrrr}
\toprule
\toprule
    \textbf{} &       & Nb    & B vs A avg & Pairs won & \multicolumn{4}{c}{Share of products (avg \%)} \\
    Chain A & Chain B & pairs & comparison & by A (\%) & A wins & B wins & Draw & Reversed \\
\midrule
    Leclerc & Geant Casino & 215   & +1.4\% & 85.1  & 61.8  & 22.4  & 15.8  & 20.4 \\
    Leclerc & Carrefour & 555   & +9.1\% & 98.4  & 78.5  & 15.1  & 6.4   & 14.7 \\
    Geant Casino & Carrefour & 89    & +7.6\% & 98.9  & 70.8  & 25.1  & 4.1   & 25.1 \\
    Carrefour & Auchan & 191   & -0.3\% & 51.8  & 46.3  & 44.3  & 9.4   & 28.9 \\
    Carrefour & Intermarche & 365   & -1.0\% & 38.6  & 45.8  & 51.2  & 3.0   & 34.0 \\
    Carrefour & Systeme U & 196   & +2.6\% & 60.7  & 57.1  & 38.8  & 4.1   & 27.3 \\
    Auchan & Intermarche & 212   & +0.8\% & 61.8  & 54.0  & 43.0  & 3.0   & 32.9 \\
    Auchan & Systeme U & 145   & +3.1\% & 66.2  & 60.5  & 35.2  & 4.3   & 27.0 \\
    Intermarche & Systeme U & 490   & +1.0\% & 51.2  & 51.5  & 41.3  & 7.3   & 25.3 \\
\bottomrule
\bottomrule
\end{tabular}
\begin{tablenotes}
      \small
      \item Among 215 pairs of Leclerc and Geant Casino competitors, Geant Casino is +1.4\% more expensive on average, and Leclerc is less expensive in 85.1\% of the pairs. On average, a Leclerc sells 61.8\% of products strictly cheaper than its Geant Casino competitor. Regardless of whether Leclerc or Geant Casino wins the overall comparison, on average, the loser i.e. most expensive store is strictly cheaper on 20.4\% of products.
\end{tablenotes}
\end{threeparttable}
\end{table}

Descriptives statics of dynamic price dispersion are reported in Table~\ref{tab:dynamic_compa_15km}. Among 114 store comparisons involving a Leclerc and a Geant Casino, 4.4\% are won by a different store in the two periods. On average, 21.2\% of products taken into account in the comparison changed order between the two periods i.e were strictly cheaper at Leclerc in first period and became strictly cheaper at Geant Casino in second period or the reverse.

\begin{table}[H]
\begin{threeparttable}
\renewcommand{\arraystretch}{0.7}% Tighter
\caption{Dynamic store level comparisons (15 km - 100 obs min)
}\label{tab:dynamic_compa_15km}
\small
\begin{tabular}{llrrr}
\toprule
\toprule
    \textbf{} &       &       & \multicolumn{2}{c}{Dynamic "Rank reversals"} \\
    Chain A & Chain B & Nb pairs & Pairs (\%) & Products (\%) \\
\midrule
    Leclerc & Geant Casino & 114   & 4.4   & 21.2 \\
    Leclerc & Carrefour & 152   & 5.9   & 24.6 \\
    Geant Casino & Carrefour & 46    & 71.7  & 42.5 \\
    Carrefour & Auchan & 49    & 42.9  & 38.0 \\
    Carrefour & Intermarche & 119   & 53.8  & 38.6 \\
    Carrefour & Systeme U & 102   & 48.0  & 37.2 \\
    Auchan & Intermarche & 86    & 22.1  & 32.4 \\
    Auchan & Systeme U & 101   & 34.7  & 29.9 \\
    Intermarche & Systeme U & 322   & 32.8  & 30.5 \\
\bottomrule
\bottomrule
\end{tabular}
\begin{tablenotes}
      \small
      \item Among 114 store comparisons involving a Leclerc and a Geant Casino, 4.4\% are won by a different store in the two periods (draws can be neglected as they virtually never happen). On average, 21.2\% of products taken into account in the comparison changed order between the two periods i.e were strictly cheaper at Leclerc in first period and became strictly cheaper at Geant Casino in second period or the reverse.
\end{tablenotes}
\end{threeparttable}
\end{table}

ADD REGRESSION RESULTS

ADD COMMENTS

\subsection{Market price dispersion}

While in the single product case, the main source of complexity attached to measuring market dispersion is the definition of markets, the presence of several products, in particular when their number is large as is the case with supermarkets, raises another major methodological issue. Absent quantity data, product choice is bound to be largely data dependent, and thus exposes results to various biases. Also, differentiation cannot be ignored by focusing on non differentiated pairs such as with competitor pair analysis, hence results are necessarily sensitive to potential misspecifications and must therefore be interpreted with great care.

The first approach followed in the paper consists in adopting markets as defined by the price comparison website. They are consequently all centered around one Leclerc store, and contain not other Leclerc store. Robustness checks are run to evaluate the impact of two potential issues: the absence of some supermarkets in the price data, and the presence of overlapping markets whenever two Leclerc are too close to each other. The first issue is addressed by narrowing the analysis to markets where virtually all active supermarkets are observed, and the second by considering that two Leclerc can be active in the same market, and alternatively by dropping overlapping markets. The second approach simply consists in considering radiuses around Leclerc stores as is standard in the literature. This typically excludes some competitors listed by Qlmc (and may include others?).

All products for which prices are available at 2/3 or more of observed supermarkets are taken into account in the analysis. Markets for which less than 100 products satisfy this criterion are dropped. Measures of price dispersion are computed both with raw prices and price residuals. Figures obtained with raw prices are likely to overestimate consumer search related price dispersion since price comparison results suggest that persistent price differences (across products and time) are non negligible. The method used to compute price residuals implies that the expected value of a large enough basket should be similar for each store in the market. As a consequence, price dispersion can be evaluated either by considering the distribution of all price residuals, or by studying separately each product distributions (which can then be aggregated to obtain a measure of price dispersion at the market level).

We investigate how market price dispersion varies with the number of sellers or the intensity of competition captured by HHI. We also investigate the link between market price dispersion and an index of market prices. The market index is built by considering how expensive each store operating in a given market is in comparison to all stores affiliated to the same chain. Alternatively, store surfaces are taken into account to weight prices indexes in the computation.

ADD DESCRIPTIVE STATISTICS

\newpage

\bibliography{references}

\newpage

\appendix

\section{Additional descriptive statistics}

\begin{table}[H]
\renewcommand{\arraystretch}{0.8}% Tighter
\caption{Overview of product families within sections}\label{tab:qlmc_families}
\small
\begin{tabularx}{\linewidth}{l >{\setlength{\baselineskip}{0.75\baselineskip}}X}
%\begin{tabular}{p{0.3\linewidth}p{0.7\linewidth}}
\toprule
\toprule
    Section & Families \\
    \midrule
    Baby and dietetic food (573) & Baby food (418); Dietetic products (155) \\
    Drinks (1,233) & Beer and Spirits (443); Fizzy drinks and Cola (244); Water (176); Juices and Smoothies (110); Squash and Cordial (101); Wine, Champagne and Cider (159) \\
    Fresh products (2,595) & Butter and Cream (199); Meat (490); Cheese (491); Milk and eggs (150); Fish (98); Delicatessen (660); Yoghurts and Chilled Desserts (507) \\
    Frozen food (368) & Ice cream and Frozen yoghurt (101); Frozen vegetables and fries (91); Frozen pizzas, pies and ready meals (128); Frozen Meat and Fish (48) \\
    Health and Beauty (2,127) & Kitchen Roll and Tissues (86); Oral care (169); Feminine care and Baby changing (138); Drugstore (97); Haircare (558); Face and body skincare (951); Men toiletries (128) \\
    Home and textile (308) & DIY and Car (9); Kitchen and dining room (50); Home Office (171); Batteries, lightbulbs and plugs (54) \\
    Household (679) & Air fresheners and insect killers (118); Laundry (124); Cloths, Gloves and Scourers (45); Cleaning (225); Dishwashing (64); Specialist laundry and Washing machine cleaner (103) \\
    Pets (239) & Cat and dog food (233); Litter (6) \\
    Savoury grocery (2,032) & Snacks (214); Condiments and Spices (609); Canned goods (406); Precooked dishes (205); Pasta, Rice and Flour (328); Soups (270) \\
    Sweet grocery (2,099) & Biscuits (294); Coffee and Tea (368); Chocolates ans sweets (450); Desserts, Sugar and Sweeteners (318); Breakfast (453); Cakes (215) \\
    Vegetables and fruits (65) & Fruits (65) \\
\bottomrule
\bottomrule
\end{tabularx}
%\end{tabular}
\end{table}


\end{document} 