\documentclass[11pt]{article}
\usepackage[T1]{fontenc}
\usepackage[utf8]{inputenc}
\usepackage{enumerate}
\usepackage{setspace}
\usepackage{amsmath,amssymb,amsthm}
\usepackage{graphicx}
\usepackage{bbm}
\usepackage[round]{natbib}
\usepackage[nohead]{geometry}
\usepackage[bottom]{footmisc}
\usepackage{indentfirst}
\usepackage{endnotes}
\usepackage{graphicx}%
\usepackage{eurosym}
\usepackage{array}
\usepackage{siunitx}
\usepackage{booktabs}
\usepackage{tabularx}
\usepackage[flushleft]{threeparttable}
\usepackage{caption}
\usepackage{subcaption}
\usepackage{rotating}
% \usepackage[hidelinks]{hyperref}
\usepackage{floatrow} %[capposition=top]
\floatsetup{footposition=bottom,capposition=top}
\renewcommand{\labelitemi}{--}
\renewcommand{\labelitemii}{$\bullet$}
\bibliographystyle{chicago}
% \geometry{left=1in,right=1in,top=1.00in,bottom=1.0in}
\let\olditemize\itemize
\renewcommand{\itemize}{
  \olditemize
  \setlength{\itemsep}{-1pt}
}

\begin{document}

\title{Elements de réponse sur les commentaires réalisés lors de la présoutenance}

\author{Etienne Chamayou}}
\maketitle

\pagebreak%
% \singlepacing

\section{Chapitre 1}

\textbf{Explication des résultats}: Ajout d'une référence à Hotelling dès l'introduction et création d'une sous-section "Discussion" qui souligne et interroge le contraste entre la réaction à la baisse qu'un modèle d'Hotelling simple laisse présager et la faiblesse des ajustements observés. L'explication que nous privilégions, sans avoir les données pour la vérifier, est que l'absence générale de réaction peut s'expliquer:
\begin{itemize}
	\item Concernant le stations relativement chères: par le fait qu'une station ne peut pas nécessairement accroitre son volume en baissant son prix du fait de sa localisation plus ou moins avantageuse ou encore à cause de son nombre de pistes.
	\item Pour les stations à bas coût: par les marges déjà faibles qui laissent a priori peu de marges de manoeuvre
	\item Dans l'ensemble par la fixation des prix au niveau des enseignes, lesquelles sont susceptibles de vouloir limiter l'hétérogénéité tarifaire pour préserver une image de marque.
\end{itemize}
Concernant les quelques hausses de prix, notre hypothèse est que certaines stations affectées ont pu basculer d'une logique de concurrence (l'idée de limiter les écarts de prix avec un ou quelques concurrents proches est revenue dans différentes discussions avec des personnes du secteur) à une logique d'extraction de rente... a priori peu tenable dans la durée. A cet égard, nous regrettons que les données disponibles sur le marché des carburants français ne permettent pas de capter les ouvertures/fermetures de stations et leur impact.

\medskip

\textbf{Groupe de contrôle}: Nous avons effectué de nombreux tests de robustesse en considérant différentes distances, régions administratives ainsi que le type de station (supermarché vs. groupe pétrolier et indépendant) sans constater d'impact significatif sur les résultats. Retenir une distance relativement courte nous paraît néanmoins nettement préférable compte tenu du fait que la localisation des stations Total Access n'est pas "représentative" de celles des stations services en général. La Table 2 montre ainsi que par rapport aux stations du groupe Total qui n'ont pas changé d'enseigne, elles ont moins  de concurrents à proximité et sont plus éloignées de la station de supermarché la plus proche. La Table 3 montre en outre qu'elles sont plus fréquemment situées en agglomération. Une distance relativement courte permet de considérer des stations opérant dans des marchés aux caractéristiques a priori relativement similaires.
Notre interprétation de la robustesse des résultats est qu'elle reflète la relative stabilité du marché. L'utilisation du carburant  comme produit d'appel par la grande distribution grande distribution (exemple de Carrefour qui garantit le prix plus bas à moins de 5 km de distance et l'interdiction de revendre à perte disciplinent en effet largement l'évolution des prix à la pompe.

\medskip

\textbf{Prolongements}: Concernant la dynamique de l'ajustement, nous avons estimé les réactions en considérant différents intervalles de temps après la bascule afin de distinguer une éventuelle phase d'ajustement (3 ou 6 mois après bascule) et un effet de long terme. A l'exception de quelques cas particuliers, nous observons très peu d'impact sur les résultats (agrégés ou par station). La relative homogénéité des baisses de prix concédées dans les stations converties, et la faiblesse des réactions des concurrents nous ont conduit à ne pas aller plus loin pour l'instant.

\medskip

\textbf{Modifications à apporter au manuscrit}:
\begin{itemize}
  \item Précision du contrôle dans la baseline? (Il s'agit de toutes les stations situées à une distance de 5 à 10 km d'une station convertie en Total Access). La table sur les diffs de diffs Total Access a été mise plus loin dans le cadre de la réorganisation de la section data décrite ci-dessous.
	\item Exercices de robustesse: à intégrer dans la version finale (en attente de retour de mon co-autheur)
	\item Réagencement: création d'une partie "Context and data" avec 3 sous parties: 
		\begin{itemize} 
		\item "The French retail gasoline market": éléments de contexts généraux avec ajout d'éléments sur les ouvertures/fermetures
		\item "Data": période et échantillon de stations, mention de l'impossibilité d'observer les ouvertures/fermetures de stations de manière satisfaisante à l'aide des données disponibles et du fait que la Table 2 donne ainsi le nombre de concurrents observés au moins une fois sur la période étudiée. Si une série de prix s'interrompt, on ne considère en effet pas que la station a nécessairement fermé. Une baisse des ventes peut en effet l'avoir soustraite à l'obligation de renseigner ses prix sur le site du gouvernement.
		\item "Definition of competition" : justification de la distance
		\item "Total Access: strategy and network development"
		\end{itemize}
	\end{itemize}
	\item Prix avec TVA et en niveau: similaire à Hastings. Pas d'impact fondamental sur nos résultats mais nous semble plus directement accessible dans une optique de vulgarisation.
	\item Page 4: « pratically » => ok
  \item Page 9: « urband » => ok
	\item Partie 4.1: On estime le traitement en utilisant toute la période car on observe des changements d'enseigne tout au long de la période étudiée, mais en effet on ne pourrait considérer que 6 mois après chaque conversion sans impact significatif sur les résultats obtenus.
	\item Dans ce paragraphe, enlever les points apres footnotes 6 et 7 dans le texte.
	\item Paragraphe 4 page 11: reprécision du contrôle: toutes les stations à plus de 5 km et moins de 10 km d'une station Total Access. Les régions considérées sont les régions administratives françaises. (un groupe de contrôle par région)
	\item Page 12 « by supermarkets » => ok
	\item Format des “reading notes” et tableau 7 trop petit => ok

\medskip

\textbf{Non couvert}:
\begin{itemize}
\item Suggestion d'utiliser l'ensemble des variables observables pour déterminer les groupes de contrôle
\item Etude de la dynamique de la réaction des concurrents en fonction de la baisse de prix implémentée par la station Total Access
\end{itemize}

\section{Chapitre 2}

\textbf{Contributions} L'introduction a été révisée de manière significative. Le second paragraphe note que les modèles de dispersion permettent potentiellement d'estimer les paramètres structurels d'un marché à partir de simples observations de prix. Ainsi, l'objectif du chapitre est de discuter leur pertinence dans le cas du marché français du carburant. La revue de litérature a été déplacée dans une section dédiée.

\medskip

\textbf{Data/Etude de cas} Les sections 2 (Contexte du marché français des carburants) et 3 (Data) ont été fusionnées et réorganisées en sous section. La sous-section market definition fournit un exemple de marché local qui essaie de répondre à la demande d'étude de cas (et reste à développer). J'ai noté pendant la présoutenance la demande concernant la possibilité de fournir des éléments sur l'historique des ouvertures / fermetures de stations. Les données disponibles dans le domaine public ne le permettent malheureusement pas (cf. footnote 9 en page 6). Dans les données brutes, les prix d'une station sont souvent fournis successivement sous différents identifiants. Après un travail important de réconciliation des identifiants, j'ai cherché si les séries de prix qui demeuraient incomplètes correspondaient à des fermetures de stations mais cela était rarement le cas. De même, j'ai observé que le démarrage tardifs de séries de prix ne correspondaient généralement pas à des ouvertures de nouvelles stations. Ceci est lié d'une part au fait que l'obligation légale de renseigner les prix ne concerne que les stations qui ont vendu plus d'un certain volume l'année précédente, d'autre part à des contrôles manifestement insuffisants. Il a ainsi fallu redresser de nombreuses erreurs dans les prix (suppression de prix manifestement trop bas ou trop élevés...).

\medskip

\textbf{Prédictions vs. observations} La section 4 (Changements de rang, lien avec les coûts de recherche) a été réorganisée en sous section dont la première explique la méthodologie proposée par Tappata, la second discute son application au marché français, la troisième fournit des statistiques descriptives, la quatrième le résultat du test, et la cinquième se consacre l'importance des chaînes en ce qui concerne les supermarchés.

\medskip

\textbf{Autres remarques}
\begin{itemize}
\item Intervention du gouvernement: la figure 10 en annexe montre la stabilité du phénomène sur la période (remarque à ajouter)
\item Choix de rester en TTC: dans la continuité de la litérature, lecture directe du point de vue du consommateur
\item Choix du diesel: carburant très majoritaire sur le marché et concernant le sans plomb on n'a nettement moins d'observations si l'on veut un carburant parfaitement comparable car il n'est pas rare qu'une station offre du SP95 et sa concurrente du SP95-E10. Ceci dit, c'est une piste de prolongement intéressante. Il s'agit du prix du diesel simple. On ne dispose pas de prix premium.
\item Les stations sur autoroute sont exclues car elle consiste un marché bien spécifique.
\item Permanent versus temporary reverse ranking? Je vais ajouter des éléments sur la fréquence des changements de rang en annexe.
\item Stratégie de chaîne: elle est mise en évidence dans le cas des supermarchés. Pour les autres stations il est clair que c'est sans doute une composante importante également mais qui elle n'est pas captée dans le travail actuel
\item Lecture table 5: footnote à ajouter? (sinon j'ai tâché de bien la décrire dans sa section...)
\item Le trend sert à capter une éventuelle tendance de long terme qui pourrait être liée par exemple au développement du l'usage du comparateur de prix
\end{itemize}

\section{Chapitre 3}

\textbf{Contributions/Rédaction} L'introduction a été réécrite pour mieux mettre en valeur les données et leur usage. La revue de literature a été déplacée dans une section "Literature and context".

\textbf{Data}: 
\begin{itemize}
\item Structure des données. La section a été modifiée pour mieux décrire les données. La rédaction souffre effectivement encore du fait qu'à l'origine je ne disposais que d'une crosse section collectée sur le site de comparaison en Mars 2015... après quoi j'ai pu récupérer une autre cross section datant de Mai 2014.
\item Homogénéité des produits: il s'agit bien toujours de produits parfaitement homogènes. Les fruits et légumes en général ne sont pas couverts par les données
\end{itemize}

\textbf{Remarques traitées mais restant à intégrer}
\begin{itemize}
\item Produits frais: 
\item Aalto et Setala (2002): dispersion et part des produits dans le budget: après investigations le nombre d'observations de prix par produit semble peu informatif dans mes données. A cet égard, il serait intéressant de réconcilier ce jeu de données avec des données qui comprennent des quantités pour avoir une mesure objective du coefficient budgétaire. J'ai regardé la dispersion sur les produits les plus vendus selon Nielsen mais cela est assez anecdotique (10 références: à mettre en annexe?) et les résultats sont très hétérogènes.
\item Rôle du drive: testé dans l'analyse des changements de rang (variable indicatrice: un magasin parmi les 2 a un drive ou les deux magasins ont un drive): pas d'effect significatif. Cela semblerait intéressant d'étudier la question de manière dynamique mais je ne suis pas sûr que mes données le permettent raisonnablement
\item Exclusion de Géant Casino dans l'analyse des déterminants de la dispersion au niveau du marché: à intégrer
\item Corrélation négative entre HHI et dispersion: cela découle de l'hypothèse qu'on doit être plutôt proche de la concurrence parfaite que du monopole mais effectivement ce n'est pas évident pratiquement ni théoriquement.
\item Prix régulier élevé vs. rabais: à intégrer 
\item Définition du HHI: l'approximation du HHI par la surface sans pondération par la distance est également réalisée. On somme bien tous les mètres carrés des différents points de vente d'une enseigne pour calculer sa part de marché.
\item Le rank reversal statique peut correspondre au résultat de Varian appliqué à différents produits ou à un cas où les prix sont corrélés négativement entre eux (MacAfee). Dans le cas statique, on pourrait néanmoins penser qu'une partie de l'hétérogénéité est liée à des différences de coûts, demande etc. L'étude de changements de rangs dynamiques est donc a priori plus pertinente (à condition d'avoir des données qui le permettent) bien que le sujet de fond soit le même: le lien entre dispersion et information des consommateurs. Rédaction 4.1 et 4.3 à modifier.
\item Product size => ok, Table 9: chiffres Carrefour Market (ok à venir dans la nouvelle table qui inclut une régression sans Géant Casino)
\item Le paramètre $\alpha$ capture la relation entre le prix moyen du produit et sa dispersion selon qu'on l'a mesure (croissante du prix du bien en valeur, captée par l'écart type, mais décroissante en termes relatifs, lorsqu'elle est mesurée par le coefficient de variation).
\end{itemize}

\end{document}