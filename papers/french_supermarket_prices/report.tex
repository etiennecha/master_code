\documentclass[11pt]{article}
\usepackage[T1]{fontenc}
\usepackage[utf8]{inputenc}
\usepackage{enumerate}
\usepackage{setspace}
\usepackage{amsmath,amssymb,amsthm}
\usepackage{graphicx}
\usepackage{bbm}
\usepackage[round]{natbib}
\usepackage[nohead]{geometry}
\usepackage[bottom]{footmisc}
\usepackage{indentfirst}
\usepackage{endnotes}
\usepackage{graphicx}%
\usepackage{eurosym}
\usepackage{array}
\usepackage{booktabs}
\usepackage{caption}
\usepackage{subcaption}
\usepackage{rotating}
% \usepackage[hidelinks]{hyperref}
\usepackage{floatrow} %[capposition=top]
\floatsetup{footposition=bottom,capposition=top}
\renewcommand{\labelitemi}{--}
\renewcommand{\labelitemii}{$\bullet$}
\bibliographystyle{chicago}
% \geometry{left=1in,right=1in,top=1.00in,bottom=1.0in}
\let\olditemize\itemize
\renewcommand{\itemize}{
  \olditemize
  \setlength{\itemsep}{-1pt}
}
\pdfminorversion=5
\pdfobjcompresslevel=3
\pdfcompresslevel=9

\begin{document}

\title{French grocery store prices: What do we know?\ \\ \ \\(Very preliminary)}
\author{E. Chamayou\thanks{e-mail:
\textit{etienne.chamayou@ensae.fr}} \\ CREST-LEI}
\maketitle

\sloppy%

\onehalfspacing

\textbf{Abstract:}

This note provides an overview of supermarket price data made available by the comparison website quiestlemoinscher.com.

\strut

\textbf{Keywords:}

\strut

\textbf{JEL Classification Numbers:} XXX

\pagebreak%
\doublespacing

\section{Introduction}

The price comparison website quiestlemoinscher.com was launched in XXX by the retail grocery store group Leclerc with a view to prove to consumers that Leclerc was cheaper than all its competitors.

\section{Overview of price records}

Data mainly consist of 13 pdf files containing prices observed at various grocery stores across France. The original purpose of these data was to create indexes reflecting the price level of the main competitors of Leclerc.

\begin{table}[H]
\renewcommand{\arraystretch}{0.7}% Tighter
\caption{Overview of price records}
\small

\begin{tabular}{llrr}
\toprule
\toprule
P  & File name                 & Ordered by                    & Price range    \\
0  & 200705\_releves\_QLMC.pdf & Rayon Famille Produit Magasin & $[0.28;45.99]$ \\
1  & 200708\_releves\_QLMC.pdf & Rayon Famille Produit Magasin & $[0.25;26.95]$ \\
2  & 200801\_releves\_QLMC.pdf & Rayon Famille Produit Magasin & $[0.15;31.52]$ \\
3  & 200804\_releves\_QLMC.pdf & Rayon Famille Produit Magasin & $[0.16;28.38]$ \\
4  & 200903\_releves\_QLMC.pdf & No order                      & $[0.36;22.23]$ \\
5  & 200909\_releves\_QLMC.pdf & Magasin Produit               & $[0.30;31.46]$ \\
6  & 201003\_releves\_QLMC.pdf & Produit Magasin (Chaine)      & $[0.30;09.99]$ \\
7  & 201010\_releves\_QLMC.pdf & Rayon Famille Produit Magasin & $[1.00;99.00]$ \\
8  & 201101\_releves\_QLMC.pdf & Rayon                         & $[0.37;09.99]$ \\
9  & 201104\_releves\_QLMC.pdf & Produit                       & $[0.31;09.99]$ \\
10 & 201110\_releves\_QLMC.pdf & Rayon Famille Produit         & $[0.16;35.02]$ \\
11 & 201201\_releves\_QLMC.pdf & Rayon Famille Produit         & $[0.15;34.99]$ \\
12 & 201206\_releves\_QLMC.pdf & Rayon Famille Produit         & $[0.16;29.14]$ \\
\bottomrule
\end{tabular}
\end{table}

Problems spotted at period level regarding prices:
\begin{itemize}
\item Period 6: it can be observed that all prices above 10.00 euros have been divided by 10 (hence min price for "Ricard pastis 45 degrés 50cl" is 1.00 euros. Since only two digits are provided after decimal point, this induces a loss in information. Furthermore, there is no general trivial fix based on spread as there are products for which all prices are above 10 euros (e.g. "Ricard pastis 45 degrés 70cl"). One way to correct the data is to compare average price across periods whenever the product is part of another record. However, after multiplication by 10, prices can not be compared directly with other periods' prices due to the loss of one digit. One may thus want to set affected prices to missing values.
\item Period 7:  it can observed that all prices below 1.00 euro have been multiplied by 100. It appears that the real highest price product within the period is "Mumm Cordon rouge champagne brut 75cl", the price of which does not exceed 35 euros so a simple fix consists in dividing all prices above that level by 100. The products thereby affected have been checked to ensure this strategy was valid (in the last resort, one could have proceeded to inter period comparisons).
\item Period 8: same issue as in period 6
\item Period 9: same issue as in period 6
\end{itemize}

\section{Overview of stores and products by period}

\begin{table}[H]
\renewcommand{\arraystretch}{0.7}% Tighter
\caption{Overview of stores and products}
\small

\begin{tabular}{rllrrrr}
\toprule
\toprule
  P &  Date start &    Date end &  Nb rows &  Nb stores &  Nb products &  Avg nb products \\
\multicolumn{6}{c}{}&  by store \\
\midrule
  0 &  09/05/2007 &  25/05/2007 &  554,691 &        344 &        2,325 &                  1,612 \\
  1 &  10/08/2007 &  31/08/2007 &  570,193 &        335 &        2,384 &                  1,702 \\
  2 &  21/01/2008 &  12/02/2008 &  544,366 &        318 &        2,374 &                  1,712 \\
  3 &  04/04/2008 &  30/04/2008 &  417,272 &        246 &        2,443 &                  1,696 \\
  4 &  01/04/2009 &  30/04/2009 &  414,911 &        701 &        1,471 &                    592 \\
  5 &  01/09/2009 &  28/09/2009 &  432,510 &        726 &        1,463 &                    596 \\
  6 &  05/03/2010 &  03/04/2010 &  446,309 &        739 &        1,466 &                    604 \\
  7 &  18/10/2010 &  16/11/2010 &  385,253 &        624 &        1,479 &                    617 \\
  8 &  28/01/2011 &  22/02/2011 &  357,188 &        634 &        1,383 &                    563 \\
  9 &  28/04/2011 &  20/05/2011 &  240,710 &        637 &          954 &                    378 \\
 10 &  17/10/2011 &  09/11/2011 &  430,968 &        640 &        1,674 &                    673 \\
 11 &  30/01/2011 &  22/02/2011 &  464,604 &        617 &        1,657 &                    753 \\
 12 &  12/05/2012 &  01/06/2012 &  607,185 &        605 &        1,805 &                  1,004 \\
\bottomrule
\end{tabular}

\end{table}

Within each period, instances of duplicates ("Produit", "Magasin") are looked for i.e. cases in which two prices are reported when only one is expected. It appears that it may be due to
 \begin{itemize}
 \item two products being reported under the same name (pack of $n$ bottles of milk of bottle bought alone, duplicates are found for many stores but not all?)
 \item two stores being reported under the same name (duplicates are found for virtually all products and only one store)
 \item ???
\end{itemize}

\begin{table}[H]
\renewcommand{\arraystretch}{0.7}% Tighter
\caption{Overview of product duplicates}
\small
\begin{tabular}{llrr}
\toprule
\toprule
P  & Nb of prod w/ duplicates & Nb prod w/ abnormal price spread   \\
0  &   2                      &  1                                 \\
1  &  10+                     &  1                                 \\
2  &   7                      &  1                                 \\
3  &   8                      &  1                                 \\
4  &   4                      &  1                                 \\
5  &   3                      &  1                                 \\
6  &   2                      & need to fix prices first           \\
7  &   4                      & need to fix prices first           \\
8  &   4                      & need to fix prices first           \\
9  &   0                      &  0 (one due to price pbms)         \\
10 &  10+                     & 1+2 suspects                       \\
11 &  10+                     & 0+1 suspect                        \\
12 &  10+                     & 0+1 suspect                        \\
\bottomrule
\end{tabular}
\end{table}

\section{Overview of product categories}

Within each period, a product label is uniquely matched with a "Rayon" and "Famille". Each "Famille" is a subcategory of a "Rayon" (really?). Categories nevertheless vary across periods.

\end{document}
