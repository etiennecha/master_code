\documentclass[11pt]{article}
\usepackage[T1]{fontenc}
\usepackage[utf8]{inputenc}
\usepackage{enumerate}
\usepackage{setspace}
\usepackage{amsmath,amssymb,amsthm}
\usepackage{graphicx}
\usepackage{bbm}
\usepackage[round]{natbib}
\usepackage[nohead]{geometry}
\usepackage[bottom]{footmisc}
\usepackage{indentfirst}
\usepackage{endnotes}
\usepackage{graphicx}%
\usepackage{eurosym}
\usepackage{array}
\usepackage{siunitx}
\usepackage{booktabs}
\usepackage{tabularx}
\usepackage[flushleft]{threeparttable}
\usepackage{caption}
\usepackage{subcaption}
\usepackage{rotating}
% \usepackage[hidelinks]{hyperref}
\usepackage{floatrow} %[capposition=top]
\floatsetup{footposition=bottom,capposition=top}
\renewcommand{\labelitemi}{--}
\renewcommand{\labelitemii}{$\bullet$}
\bibliographystyle{chicago}
% \geometry{left=1in,right=1in,top=1.00in,bottom=1.0in}
\let\olditemize\itemize
\renewcommand{\itemize}{
  \olditemize
  \setlength{\itemsep}{-1pt}
}

\begin{document}

\title{Elements de réponse sur les commentaires réalisés lors de la présoutenance}

\author{Etienne Chamayou\thanks{e-mail:
\textit{etienne.chamayou@ensae.fr}}\medskip\\{\normalsize CREST and Department of Economics, Ecole Polytechnique }}
\maketitle

\pagebreak%
\doublespacing

\section{Chapitre 1}

Nous avons estimé les réactions en considérant des intervalles de 3 mois et 6 mois, et, à l'exception de quelques cas particuliers, observons très peu de variations dans les résultats selon les périodes considérées.
De même sur les groupes de contrôle (suggestions d'utiliser l'ensemble des variables observables), nous avons effectué de nombreux tests de robustesse en considérant différentes distances et en conditionnant sur le type de station (supermarché vs. groupe pétrolier et indépendant) sans constater d'impact significatif sur les résultats.
Globalement, les analyses confirment la relative stabilité des prix en longue période, y compris en présence de chocs a priori susceptibles d'engendrer des réactions significatives. Ce résultat justifie en quelque sorte l'approche du second chapitre qui suppose qu'on observe un marché globalement à l'équilibre et que l'on peut décomposer la politique de prix d'une station en une composante de long terme (un effet fixe station) et des fluctuations de courte durée.

L'explication que nous privilégions, sans avoir les moyens de la vérifier de manière quantitative, est que l'absence générale de réaction peut s'expliquer:
\begin{itemize}
\item Concernant le stations relativement chères: par le fait qu'une station ne peut pas nécessairement accroitre son volume en baissant son prix du fait de sa localisation plus ou moins avantageuse ou encore à cause de son nombre de pistes
\item Pour les stations à bas coût: par les marges déjà faibles qui laissent a priori peu de marges de manoeuvre
\item Dans l'ensemble par la fixation des prix au niveau des enseignes, lesquelles sont susceptibles de vouloir limiter l'hétérogénéité tarifaire pour préserver une image de marque.
\end{itemize}

Je retiens la suggestion d'étudier la réaction ainsi que sa dynamique en fonction du niveau initial du prix, mais n'ai pu l'intégrer.

Concernant les modifications à apporter au manuscrit:
\begin{itemize}
\item Le

\section{Chapitre 2}



\section{Chapitre 3}


\end{document}