\documentclass[english]{article}
\usepackage[T1]{fontenc}
\usepackage[latin1]{inputenc}
\usepackage{enumerate}
\usepackage{setspace}
\usepackage{amsmath,amssymb,amsthm}
\usepackage{graphicx}
\usepackage{bbm}
\usepackage[round]{natbib}
\usepackage[nohead]{geometry}
\usepackage[bottom]{footmisc}
\usepackage{indentfirst}
\usepackage{endnotes}
\usepackage{graphicx}%
\usepackage{eurosym}
\usepackage{array}
\usepackage{booktabs}
\usepackage{caption}
\usepackage{subcaption}
% \usepackage[hidelinks]{hyperref}
\usepackage{floatrow} %[capposition=top]
\floatsetup{footposition=bottom,capposition=top}
\renewcommand{\labelitemi}{--}
\renewcommand{\labelitemii}{$\bullet$}
\bibliographystyle{chicago}
% \geometry{left=1in,right=1in,top=1.00in,bottom=1.0in}
\let\olditemize\itemize
\renewcommand{\itemize}{
  \olditemize
  \setlength{\itemsep}{-1pt}
}

\begin{document}

\title{French grocery store prices: Evidence from a price comparison website\ \\ \ \\(Very preliminary)}
\author{E. Chamayou\thanks{e-mail:
\textit{etienne.chamayou@ensae.fr}} \\ CREST-LEI}
\maketitle

\sloppy%

\onehalfspacing

\textbf{Abstract:}

The French supermarket chain Leclerc operates a price comparison website which allows to compare each of its stores with some local competitors, and provides aggregate chain comparisons at the national level. This papers questions the contribution of this website to consumer information and investigates heterogeneity in local market competitiveness across France.

\strut

\textbf{Keywords:}

\strut

\textbf{JEL Classification Numbers:} D83, L81, M31

\pagebreak%
%\doublespacing

\section{Introduction}

Since the seminal paper of \cite{STI61}, a large literature on the topic of consumer search and price dispersion has developed. Following \cite{VAR80}, a paradigm has emerged in which price dispersion results from price randomization by sellers in equilibrium. Empirical research has long been hampered by the scarcity of appropriate data. Recent examples include \cite{TAP11} and \cite{CHA15} which measure and investigate local price dispersion in retail gasoline markets. Distance between sellers is used as proxy for consumer information, whose link with price dispersion is shown to be supported by available data.

Despite the complexity induced by the multiplicity of products, competition between retail grocery stores appears to be a natural candidate for empirical investigations on the role of consumer information. \cite{LAC02}, using store-level monthly prices of four homogeneous products sold in Israel, has documented the persistence of price dispersion over time, with sellers changing ranks in price distributions in a way that is consistent with \cite{VAR80} predictions. Somewhat surprisingly, very few papers have sought to replicate and challenge the findings with more comprehensive data. A recent example is \cite{DUB15}, which yet also focuses on four products sold in French supermarkets.

Using price data obtained from the French price comparison website quiestlemoinscher.com, this paper documents static and dynamic price dispersion between French retail grocery stores. The first part of this paper details the data through a review of the methodology used by the website to compare supermarkets. Store comparisons are shown to be actually often highly sensitive to the basket of goods considered, suggesting that it is very costly for consumers to purchase at the lowest price. TODO: comment results + add public policy conclusions

\section{Supermarket comparison on Quiestlemoinscher.com}

\subsection{French retail grocery stores}

As of 2015, the French retailing industry was dominated by six firms, which accounted for over 80\% of total sales: Carrefour and Leclerc were the two largest with c. 20\% market shares each, followed by Intermarche (14\%), Casino (12\%), Auchan (11\%) and Systeme U (10\%). An important distinction between firms lies in the ownership structures: while Carrefour, Casino and Auchan own most of the stores operated under their brands, Leclerc, Intermarche and Systeme U are essentially franchise networks.

The creation of the comparison website Quiestlemoinscher.com is part of a long term strategy of the group Leclerc to prove the competitiveness of its store prices. Soon after the launch in May 2006, the website was forced to close by a court decision, following a complaint by Carrefour regarding the lack of transparency and potential biases in store and product choices. An updated version of the website was released on November 2006 and has since then remained online. Legal proceedings nervetheless continued until the rejection by the court of cassation of Carrefour's claims in January 2010.

A major merit of the legal action undertaken by Carrefour is its consequence for the transparency, namely the release of well identified store product price data. The following section provides an overview of the methodology of the comparison website, two crucial aspects of which are the choices of stores and products.

\subsection{Products}

Before 2013, the website only offered comparisons at the chain level. For each chain, according to Qlmc, prices were collected a sample of store expected to be representative of the store population. Broad constraints were thus imposed on store location and size, while exact store choice was claimed to be random.

From 2013 on, the development of the "drive" concept in France has allowed the comparison website to cover far more stores, and thus to start displaying store level comparison. The concept of "drive" implies that consumers are offered the opportunity to shop online from a physical store (at the same prices) and collect their purchases whenever it suits them. The collection of prices can then be achieved efficiently on the internet, as opposed to traditional store visits. As of March 2015, Qlmc claimed to cover 60\% of the stores of the 10 supermarket chains compared (44\% in August 2013).

Regarding store level comparisons, Qlmc  methodology states that each Leclerc is compared with a selection of its most relevant competitors within 30 km, based on Leclerc managers' expertise. The website also indicates that stores whose surface is smaller than 1,000 $m^2$ are excluded, as well as stores belonging to chains which are deemed to be too differentiated (e.g. discounters). Finally, Leclerc stores are not included among potential competitors.

As of March 2015, only national brand products are covered by the website. Products are identified at the bar code level. There are seven food product categories: meat and fish, vegetables and fruits, backery, fresh food, frozen food, savoury grocery, sweet grocery, baby food and drinks. Non food products are split in four categories: health and beauty, houdesold, pets and home and textile. Products are further classified wihin product families. The number of products covered within each family is determined by the volume of national hypermarket and supermarket sales, with a global objective of 3,000 products. Within each family, products are chosen based on the national hypermarket and supermarket detention rate. Products whose detention rate is below 30\% (i.e. products referenced by less than 30\% of the stores) are dropped. This led to a total of 2,461 national brand product references covered for March 2015 (2,510 in August 2013).

\subsection{Price comparison}

The comparison of Leclerc with a competing chain can be described in a few simple steps. First, the average price of each product is computed successively over Leclerc stores and over the competing chain's stores. Whenever a product is carried by too few stores of one of the two chains, the product is dropped from the comparison. The basket of goods compared is then simply composed by all products for which an average price is available for both chains.

The comparison result displayed by Leclerc is then the percentage difference between the price of the basket for the competing chain and for Leclerc:
\begin{align*}
\frac{\sum\limits_{i} P_{iC} - \sum\limits_{i} P_{iL}}{\sum\limits_{i} P_{iL}}
\end{align*}s
where $i$ refers to all products in the baskets, $P_{iC}$ and $P_{iL}$ respectively stand for the average price of product $i$ for the competing chain ($C$) and for Leclerc ($L$). The comparison between two stores is very similar except that it uses store prices instead of average chain prices.

\subsection{Data and replication}

This section provides an overview of the data collected from the website in 2015 and of the replication of its results. Potential biases in terms of store and product selection are discussed, and the robustness of results is checked with 2007-2012 data. In March 2015, the website was explored with a view to find the best way to extract price data. The only feasible solution appeared to successively crawl comparisons between Leclerc stores and each of their competitors. This implies that thereby obtained data are bound to be significantly less exhaustive than the full price database used by Qlmc to establish chain level comparisons.

\subsection{Leclerc competitors}

A total number of of 575 Leclerc stores were found to be listed on the website. The comparison of each store with its respective selection of competitors yielded 2,390 pairs of stores, involving 1,815 unique Leclerc competitors (the number of pairs is larger since a store can be listed as a competitor of several Leclerc stores). Data were missing for 14 Leclerc stores and 51 Leclerc competitors. This implies that among competitors of the 561 Leclerc stores for which price data have been collected, 36 out of 1811 are missing ($\le 2 \%$).

Table~\ref{tab:qlmc_comp} provides an overview of competition according to qlmc comparisons, even though it must be reminded that the website does not claim to be comprehensive. On average, a Leclerc store is compared with 5 competitors, and over 50\% over the stores have a competitor listed within 2 km. The furthest competitor is generally within 30 km (15 km for almost half of them), except for 28 stores. Also, 14 Leclerc stores have no competitor listed within 10 km. If one store met these two criteria (or variations), it would be reasonable to suspect that the lack of nearby competitors led to include competitors beyond reasonable distance but it is not the case. For instance, the Leclerc outlet which has the furthest competitor in the data (67 km) is listed with 7 competitors, of which 5 are located within 30 km.

\begin{table}[H]
\renewcommand{\arraystretch}{0.7}% Tighter
\caption{Overview of competition around the 575 Leclerc stores in Qlmc}\label{tab:qlmc_comp}
\small
\begin{tabular}{lr|rrrr}
\toprule
\toprule
{}         & Nb          &    \multicolumn{4}{c}{Distance (km) to} \\
{}         & competitors &    closest & furthest & mean & median \\
\midrule
Mean  &   5.0 &   2.4 &  15.9 &   8.8 &     8.5 \\
Std   &   1.6 &   2.5 &   9.7 &   5.1 &     6.0 \\
Min   &   1.0 &   0.1 &   0.9 &   0.8 &     0.5 \\
Q10   &   3.0 &   0.7 &   4.6 &   3.0 &     2.5 \\
Q25   &   4.0 &   1.1 &   8.4 &   4.8 &     3.7 \\
Q50   &   5.0 &   1.8 &  15.3 &   7.8 &     6.5 \\
Q75   &   6.0 &   2.7 &  21.5 &  12.3 &    12.5 \\
Q90   &   7.0 &   4.7 &  26.3 &  15.7 &    18.0 \\
Max   &  12.0 &  21.1 &  67.0 &  28.6 &    28.5 \\
\bottomrule
\bottomrule
\end{tabular}
\end{table}

TODO: add paragraph about competitor selection (share of competitors compared vs. all within 30 km etc.)

Table~\ref{tab:qlmc_chain_repr} provides an overview of store coverage for the ten national chains compared on qlmc. Coverage is high and rather close to coverage in the full qlmc sample for chains which are characterized by large store surfaces: Auchan, Carrefour, Cora, Geant and Leclerc. This can be explained by the fact that Leclerc is present across all regions and operates rather large stores. Regarding chains with smaller store formats, coverage is lower in percentage and also generally with respect to the full qlmc sample (e.g. 22\% for Casino vs. 39\% claimed by the website). Two natural explanations are the fact that stores from these chains are generally less likely to be considered relevant local competitors for Leclerc stores, and the slower development of "drive" within smaller stores (which make price collection less costly).

\begin{table}[H]
\renewcommand{\arraystretch}{0.7}% Tighter
\caption{Representation of major national chains}\label{tab:qlmc_chain_repr}
\small
\begin{tabular}{lrrr}
\toprule
{} &  \multicolumn{2}{c}{Nb stores}   &  Coverage      \\
{} &   France (LSA) & Extract QLMC    & (\%)           \\
\midrule
AUCHAN           &            142 &             112 &            79 \\
CARREFOUR        &            222 &             172 &            77 \\
CARREFOUR MARKET &            925 &             242 &            26 \\
CASINO           &            392 &              85 &            22 \\
CORA             &             58 &              54 &            93 \\
GEANT            &            107 &              93 &            87 \\
INTERMARCHE      &           1770 &             541 &            31 \\
LECLERC          &            579 &             572 &            99 \\
SIMPLY MARKET    &            305 &              52 &            17 \\
SYSTEME U        &           1030 &             420 &            41 \\
\bottomrule
\bottomrule
\end{tabular}
\end{table}

\subsection{Products}

TODO: Table by family... compare with 2007-2012 split (and 2014), include product examples (?),nb products by chain stores (large only vs small?)

\subsection{Robustness checks with 2007-2012 data}

Price records of the 2007-2012 period were made available by Qlmc in pdf files. They were thus downloaded and processed to build a database (footnote: pdf files exist for price records of May 2014 but are not used in the current version of the paper).

\begin{table}[H]
\renewcommand{\arraystretch}{0.7}% Tighter
\caption{Overview of stores and products}
\small

\begin{tabular}{rllrrrr}
\toprule
\toprule
  P &  Date start &    Date end &  Nb rows &  Nb stores &  Nb products &  Avg nb products \\
\multicolumn{6}{c}{}&  by store \\
\midrule
  0 &  09/05/2007 &  25/05/2007 &  554,691 &        344 &        2,325 &                  1,612 \\
  1 &  10/08/2007 &  31/08/2007 &  570,193 &        335 &        2,384 &                  1,702 \\
  2 &  21/01/2008 &  12/02/2008 &  544,366 &        318 &        2,374 &                  1,712 \\
  3 &  04/04/2008 &  30/04/2008 &  417,272 &        246 &        2,443 &                  1,696 \\
  4 &  01/04/2009 &  30/04/2009 &  414,911 &        701 &        1,471 &                    592 \\
  5 &  01/09/2009 &  28/09/2009 &  432,510 &        726 &        1,463 &                    596 \\
  6 &  05/03/2010 &  03/04/2010 &  446,309 &        739 &        1,466 &                    604 \\
  7 &  18/10/2010 &  16/11/2010 &  385,253 &        624 &        1,479 &                    617 \\
  8 &  28/01/2011 &  22/02/2011 &  357,188 &        634 &        1,383 &                    563 \\
  9 &  28/04/2011 &  20/05/2011 &  240,710 &        637 &          954 &                    378 \\
 10 &  17/10/2011 &  09/11/2011 &  430,968 &        640 &        1,674 &                    673 \\
 11 &  30/01/2011 &  22/02/2011 &  464,604 &        617 &        1,657 &                    753 \\
 12 &  12/05/2012 &  01/06/2012 &  607,185 &        605 &        1,805 &                  1,004 \\
\bottomrule
\end{tabular}
\end{table}

\bibliography{references}
\bibliographystyle{chicago}

\end{document}
