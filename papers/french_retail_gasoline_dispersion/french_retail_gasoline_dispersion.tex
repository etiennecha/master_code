\documentclass[english]{article}
\usepackage[T1]{fontenc}
%\usepackage[latin1]{inputenc}
\usepackage{enumerate}
\usepackage{setspace}
\usepackage{amsmath,amssymb,amsthm}
\usepackage{graphicx}
\usepackage[round]{natbib}
\usepackage[nohead]{geometry}
\usepackage[bottom]{footmisc}
\usepackage{indentfirst}
\usepackage{endnotes}
\usepackage{graphicx}%
\usepackage{eurosym}
\usepackage{array}
\usepackage{booktabs}
\usepackage{siunitx}
\usepackage{dcolumn}
\newcolumntype{.}{D{.}{.}{-1}}
\usepackage{caption}
\usepackage{subcaption}
\usepackage{tabularx}
\usepackage[flushleft]{threeparttable}
% \usepackage[hidelinks]{hyperref}
\usepackage{floatrow} %[capposition=top]
\usepackage{bbm}
\floatsetup{footposition=bottom,capposition=top}
\renewcommand{\labelitemi}{--}
\renewcommand{\labelitemii}{$\bullet$}
\bibliographystyle{chicago}
% \geometry{left=1in,right=1in,top=1.00in,bottom=1.0in}
\let\olditemize\itemize
\renewcommand{\itemize}{
  \olditemize
  \setlength{\itemsep}{-1pt}
}

\begin{document}

\title{Price dispersion on the French retail gasoline market\ \\ \ \\Working paper}
\author{Etienne Chamayou\thanks{e-mail:
\textit{etienne.chamayou@ensae.fr}}\medskip\\{\normalsize CREST and Department of Economics, Ecole Polytechnique }}
\maketitle

\sloppy%

\onehalfspacing

\textbf{Abstract:}
Using a large panel of daily French diesel prices, this paper finds support for a relation between price dispersion and imperfect consumer information. The volatility in price rankings between pairs of competitors is indeed found to be positively correlated with the distance that separates them, namely a measure of consumer search costs. Furthermore, price dispersion is strongly connected with price levels. Pairs of supermarket competitors, which operate at relatively low markups, exhibit less dispersion than pairs of independent or oil company gas stations. At the market level, a higher diesel cost is associated with lower dispersion, and a higher number of nearby competitors is associated with increased dispersion for high markup gas stations. These results suggest that supermarkets compete for a well informed and thus highly price sensitive demand, while oil company and independent gas stations generally address customers characterized by more loyalty or higher search costs.

\strut

\textbf{Keywords:} Consumer search, Price dispersion

\strut

\textbf{JEL Classification Numbers:} L13

\pagebreak%
%\doublespacing

\section{Introduction}

The understanding of retail gasoline prices has motivated a rich academic literature, marked by concerns of insufficient competition and collusion. The first commonly cited paper of the literature, \cite{BAC91}, was meant to formally investigate an hypothesis of the British Monopolies and Mergers Commission, according to which gas stations might take advantage of cost fluctuations, adjusting prices downwards at a slower pace than upwards. Using a partial adjustment model with fortnightly UK prices between 1982 and 1989, the paper found support for the "rocket and feather" asymmetry hypothesis. \cite{BOR97}, working with weekly US data over a six year period ending in 1992 and an Error Correction Model, also found evidence of such a phenomenon, in particular between city branded terminal prices and retail gasoline prices. Empirical investigations on competition at the gas station-level have somewhat lagged behind, most likely due to data limitations. \cite{HAS04}, studying a merger in California, found that independent gas stations fostered competition, while an increase in the share of branded gas stations was associated with higher price levels. More closely connected to the "rocket and feather" literature, several papers have investigated price dynamics at the city or gas station level and found evidence of "Edgeworth cycles", namely cycles unrelated to cost variations (\cite{ECK02}, \cite{ECK03}, \cite{ECK04a}, \cite{ECK04b}, \cite{NOE07a}, \cite{NOE07b} and \cite{NOE08} in Canada, \cite{LEW09} and \cite{LEW11a} in the US).

The complexity and diversity of observed price patterns have led to question competition intensity and its determinants. \cite{STI61} has initiated a large body of literature by highlighting the link between the "ignorance in the market", namely a lack of consumer information, and price dispersion i.e. the persistence over time of different prices for a homogeneous good. The retail gasoline market is an interesting candidate when it comes to studying the impact of consumer search costs on competition. Consumers indeed purchase only one relatively homogeneous product and typically face significant costs to remain informed about prices.

\cite{BAR04} were the first to investigate price dispersion in the retail gasoline market, using a data set of nearly 3000 gas station prices within four US areas on a single day, in 1997. The non observation of price dynamics implies a limited ability to control for the impact of station-specific characteristics on prices, and the necessity to consider both static and dynamic theoretical explanations of price dispersion\footnote{In a theoretical context, static price dispersion reflects the use of heterogeneous pure price strategies, while dynamic price dispersion typically refers to price randomization by sellers in equilibrium.}. Under monopolistic competition, price dispersion related to heterogeneity in seller demand or cost should decrease when seller density increases, and so should the average price. Under a search-theoretic approach, the average price can either decrease or increase\footnote{It decreases in Carlson and McAfee (1983), in which price dispersion is static, and increases in \cite{VAR80}, which has dynamic price dispersion}, but  seller density and price dispersion should be negatively correlated. This effect can yet be mitigated or reinforced depending whether seller density influences search costs. In particular, \cite{VAR80} finds that a higher proportion of informed customer can lead to an increase or decrease in the variance of prices, depending on the model's parameters. \cite{BAR04} measure the density of sellers by the number of gas stations within a 1.5-mile radius around each station. Price dispersion is measured by unexplained variations in prices, namely the squared residuals of the regression of the log of prices on market characteristics, including seller density. An increase in the number of nearby gas stations is found to be associated with a reduction in price dispersion.

\cite{HOS08} provide some insights about price dispersion\footnote{They focus on the explanation of gas station mark up levels, the main determinant of which is found to be brand affiliation, and observe many changes in mark up levels  on a yearly basis} with weekly prices from 272 gas stations around Washington DC  between 1997 and 199. They first regress prices on week time indicators, common to all gas stations, and use the residuals to study the persistence of gas station pricing policies. They then add station fixed-effects to the regression so that residuals reflect deviations from each station’s typical price level. Controlling for station fixed-effects accounts for much of the persistence in prices, meaning that a significant amount of dynamic price dispersion is observed once gas station long term pricing policies are taken into account.  The data and method employed offer an improvement over \cite{BAR04} as they shed light on dynamics which require to go beyond models of static price dispersion, but they don’t allow to provide an order of magnitude and study variations across markets.

\cite{LEW08} reconciles the two previous approaches by using station level fixed-effects to control for differentiation, and investigating the relationship between price dispersion and local market characteristics. Data include price records of 327 gas stations in the San Diego area on each Monday in 2000 and 2001 (91 weeks). The paper finds a negative relationship between seller density and price dispersion, in line with Barron, Taylor and Umbeck (2004), and refine this result by introducing a distinction between high-brand groups, composed by premium branded stations, and low-brand groups, which include discount brand and unbranded stations. The relationship between the density of low-brand sellers and price dispersion is found to be negative, while high-brand sellers have a weakly positive or insignificant impact. Lewis (2008) however observes that a more localized measure of dispersion* can lead to find a positive relationship between density and price dispersion, which suggests a complex relationship between seller heterogeneity and price dispersion.

Finally, \cite{TAP11} make two significant contributions to the literature. They introduce a formal test regarding the relationship between price dispersion and consumer search, using distance between competing gas stations as a proxy for consumer information, and then use price dispersion measured at the market level to investigate the relationship between price dispersion and market characteristics.

The following paper elaborates on the methodology used by \cite{TAP11}. A remarkable specificity exhibited by the French market is the presence of significant static price dispersion, with supermarket gas stations setting prices generally 8 to 10 euro cents per liter cheaper than oil company and independent gas stations. Findings support a connection between consumer search and price dispersion. Among pairs of competing stations, price dispersion is indeed found to significantly increase with distance separating gas stations, namely a measure of information imperfection. Price dispersion is higher when competitors are both independent or oil company gas stations than when they are operated by supermarkets. This suggests that supermarkets generally address a well informed, thus more price sensitive demand, justifying the lower observed markups. At the market level, price dispersion is found to increase with the number of competitors and decrease with price.

A major contribution of the consumer search literature was made by \cite{VAR80} through the modeling of price dispersion as a result of mixed pricing strategies. According to the paper, price dispersion thereby obtained can be interpreted as "temporal" price dispersion, typically in the form of "sales". This provides a rationale for rank reversals i.e. a seller being either cheaper or more expensive than a competitor, both with positive probabilities.

The dynamic interpretation of the model is not unambiguous however. In the model, firms are ex-ante indifferent between all prices in the support of the equilibrium price distribution (also holds in terms of randomization over utilities). Ex-post, indifference obviously no more exists. The cheapest firm attracts shoppers but would be better off increasing its price to (almost) match the second cheapest price. Other firms would rather increase their price to consumers' reservation price, or try to undercut the cheapest firm. In the retail gasoline market, while it may not be possible for firms to change prices on too frequent a basis, periods of significant oil price variations reveal that gas stations can adjust prices on a daily basis. If one admits that gas stations play according to \cite{VAR80} on a given day, it is thus not clear why sellers would wish to keep prices unchanged the following day, in contradiction with observed price rigidities. A possible explanation may be that firms refrain from changing prices too often for fear of triggering more search by consumers and thus more intense competition. Whether this can be obtained in a fully competitive setting  or would require some kind of collusion (possibly at the retail chain level or at the local level) remains an open question.

\cite{TAP11} study price dispersion with daily prices of gas stations within four states in the US over one year and a half. In order to test the relation between consumer information and price dispersion, they rely on the assumption that distance between sellers is likely to reflect consumer information. Indeed, when gas stations are located in the same street, a higher share of consumers is likely to perfectly observe prices than when gas stations are located further away from each other. When the share of uninformed consumers becomes negligible, sellers can be expected to compete a la Betrand (or Hotelling) and prices should essentially match cost fluctuation except for some frictions. On the other hand, if information is imperfect, persistent dispersion can be expected to arise following the intuition exposed in \cite{VAR80}.

\cite{TAP11} measure temporal price dispersion between two stations as the probability that the gas station which is in general cheaper (in terms of day count) turns out to be more expensive. Formally, considering the prices $p_{it}$ and $p_{jt}$ of two stations $i$ and $j$ over $T_{ij}$ days, such that $p_{it} \ge p_{ij}$ is observed most of the time, the rank reversals statistic writes:
\begin{align*}
r_{ij} = \frac{1}{T_{ij}} \sum_{t=1}^{T_{ij}} \mathbbm{1}_{p_{jt} > p_{it}}
\end{align*}
Alternatively, in order to check the robustness of results to the influence of persistent price differences between gas stations, they use standard deviation of price differences to account for pair prices dispersion:
\begin{align*}
\sigma_{ij} = \sqrt{\frac{1}{T_{ij}} \sum_{t=1}^{T_{ij}} [s_{ijt} - \bar{s}_{ij}]^2}
\end{align*}
They find temporal price dispersion to be positively correlated with distance, which is all the more convincing as close gas stations tend to be exhibit lower persistent price differences. Regressions of various measures accounting for price dispersion on marginal costs and the number of firms in the market (built by considering circles of varying radiuses) yields results consistent with the extension of \cite{VAR80} proposed by the paper (resp. negative and positive impacts).

\section{The French retail gasoline market}

\subsection{Retail gasoline distribution}

Diesel consumption currently accounts for c. 80\% of retail gasoline sales in France. The share of diesel in total gasoline consumption has kept increasing over the last decades, largely as a result of a lower tax. Meanwhile, the size of the French retailer network has decreased at a steady pace, from c. 40,000 gas stations in 1980 to c. 12,000 currently.  Unlike most other European countries, the French market is characterized by a strong competitive pressure generated from gas stations operated by supermarket chains. They currently represent c. 50\% of sales in retail gasoline.

According to the French Union of Petroleum Industries (UFIP), there were 11,662 gas stations operating in France (of which 4,947 operated by supermarket chains) in 2012 and 11,476 in 2013 (4,979 for supermarkets). Regarding volumes, it was reported that 1,506 gas stations sold less than 500m$^{3}$ in 2012 (1,433 in 2013), with the median gas station selling between 1,000 and 3,000m$^{3}$ (same for 2013). As of May 20, 2014, the price comparison website Zagaz listed 12,832 gas stations, but no price had been recorded for long for many of them. This figure can thus be considered as an upper bound of the actual number of gas stations.

Gas stations are essentially either owned and operated by a chain or with a "location-gerance" contract according to which the manager receives a commission on gasoline sold (e.g. Total SA, the largest gas station operator, has reported that only 200 gas stations set prices independently among the c. 2,300 gas stations of its "Total" chain). Industry margins are widely acknowledged to have decreased significantly over the last decade, as a result of competition by supermarket chains and increasingly stringent environmental regulations. This has led some oil companies to exit the market (Shell and BP) or to reduce significantly the size of their network (Esso, Total).

Key cost components are the cost of wholesale gasoline, including delivery fees,  gas station operating expenses, and taxes. Taxes included a fix part called TICPE, which slightly varies between regions, and the classical Value-Added Tax (19.6\% over the period studied, which bear on cost and TICPE).

At an aggregate level, two kinds of consumers can be distinguished: businesses and individual customers. Businesses are typically offered card programs which allow them to monitor employees' consumptions and obtain rebates. An important implication is that the price of the gas station is irrelevant (or only partly relevant) to a significant number of transactions in the market. Individual consumers pay the posted price, and can get information from a variety of sources: at gas stations, on their gps, mobile phone applications (e.g. Zagaz, Carbeo, Essence Free) and on a computer or mobile phone browser (Prix-Carburants.gouv.fr).

The period covered by the paper is marked by two significant events of different natures. The first is the creation by the largest gas station operator in the country, Total, of a discount brand with a view to recapture market shares lost to supermarket gas stations. This creation was achieved through the rebranding of c. 600 gas stations between 2011 and 2014, accompanied for about half of them by a c. 10 euro cents per liter decrease in prices. The second event is of political nature. On August 29, 2012, following an election promise made by Francois Hollande, the government announced a decrease in price of 6 euro cents per liter, (to be) achieved by a decrease of tax of 3 Euro cents per liter and an equivalent "effort" by gas station operators.

\subsection{Price comparison}

Since 2007, French gasoline retailers are required by law to keep prices updated on the price comparison website prix-carburants.gouv.fr. Small gas stations%
\footnote{Stations having sold over 500m$^{3}$ gas the previous year%
} are exempted from this obligation hence c.10,000 gas stations are observed out of an estimated total number of 12,000 retailers%
\footnote{A 2012 governmental report on the French retail gasoline market notes that "nobody knows precisely the number of gas stations operating in the markets". Two other comparison websites, carbeo.com and zagaz.com, were created in 2005 and 2006, relying on user provided information. Zagaz has stuck to its "crowdsourcing" philosophy until 2014, while Carbeo started purchasing licences from the government in 2009. In 2012, the governmental body in charge of town and country planning worked with Zagaz data to study the French retail gas station network.%
}.
Significant limitations of the governmental comparison website include the fact that users have never been provided a way to report errors such as wrong or out of date prices or wrong gas station locations, and that price comparison functionalities have always remained very poor: e.g. one cannot view prices for a given highway, nor the rivals of a given gas station on a map. As a consequence, it does not seem far-fetched to suspect that the creation of this website was actually detrimental to consumer information, as it diverted drivers from other comparison websites such as Zagaz at a time when they crucially needed to grow their user base.

\subsection{Descriptive statistics}

Prices and brand changes were collected from prix-carburants.gouv.fr on a daily basis between September 4, 2011 and December 4, 2014, hence a period of about 3 years, interrupted by some gaps related to data acquisition issues (Figure~\ref{fig:brent_and_diesel}) . Over this period, prices of 10,180 gas stations (after duplicate reconciliations) were recorded, of which 437 were located on highways, 124 on the island of Corsica, and 402 were found to have insufficient or suspicious price data. The analysis was thus performed with a total number of 9,217 gas stations. Due to the fact that price observations are not always available (gas station maintenance, probable mistakes, stations which are not required by law to keep prices posted), there is actually an average of 7,895 prices observed on a daily basis. On average, c. 1,500 gas stations (c. 18\% of gas stations observed and retained) change prices within a day. The average gas station changes price a little less than every week.

\begin{figure}[H]
    \caption{Daily Brent and French diesel retail prices}
	\centering
		\includegraphics[width=\textwidth]{graphs/macro_trends.png}
\label{fig:brent_and_diesel}
\end{figure}

Table~\ref{tab:station_chains} provides an overview of retailer chains and prices on the last day of the period studied. Except for two discount chains operated by oil companies, supermarket gas stations are found to set prices significantly lower than these of oil company and independent gas stations. Within supermarket groups, chain appear to provide a somewhat more detailed signal. For instance, the average gas station operated by a Carrefour (hypermarket) has a diesel price of 1.15 euros per liter vs. 1.18 for a Carrefour market (large supermarket) and 1.20 for a Carrefour contact (small supermarkets, often located in city centers). The average difference with a Total gas station thus varies from 12 to 7 euro cents per liter depending on the chain within the Carrefour group. These observations are essentially robust to the considered day. A noteworthy change is the creation by Total S.A. of a new chain, Total Access, with a view to expand its discount offer. As a consequence, the previous discount chain of Total S.A., Elf, disappears over the period and is replaced by Total Access. The new discount chain also includes c. 300 former Total stations for a which the conversion operation is accompanied by a significant change in pricing policy. The impact of this operation is studied in a companion paper, \cite{CHA16}.

Table~\ref{tab:station_stats_des} contains some information about the heterogeneity observed between gas stations. Supermarket gas stations change prices on a more frequent basis than others, even though the value of changes does not appear to differ significantly. Their closest rival is generally further (2.6 vs. 1.8 km), and the number of nearby competitors higher (2.5 vs. 4.1 competitors). The location of gas stations also accounts for significant heterogeneity. Gas stations within and around Paris tend to change prices for frequently, with smaller variations, and are closer to their closest rival than average.

Regardless of the specifications used to account for gas station and local market characteristics, chains (or types reported in Table~\ref{tab:station_chains}) retain a major predictive power on price. Indeed, despite differences reported between supermarket gas stations and others in Table~\ref{tab:station_stats_des}, more than half non supermarket gas stations are located within 1.4 km from a gas station operated by a supermarket, hence local market characteristics are not systematically different.

\section{Price dispersion and consumer information}

A database is built in which observations are pairs of stations assumed to compete within the same market. In the following analysis, distance as the crow flies is used with an upper distance limit of 3km (figures are provided for a limit of 5km in appendix). This results in a database describing price competition between 12,264 pairs of rival gas stations (respectively 25,007 pairs with 5km).

The treatment of persistent differences in average prices is a sensitive issue in the empirical analysis of dispersion. It has been frequently addressed in the literature by working with price residuals i.e. after controlling for seller (and occasionally time) fixed effects. However, beyond the practical errors inherent to the statistical treatment, this approach is only valid to the extent that static price dispersion mirrors heterogeneity in offered utility, common to all consumers. In the case of French gasoline retailers, the relatively low predictive power of observed seller characteristics regarding prices does not offer strong support for this hypothesis. As a consequence, the analysis focuses on pairs of gas stations which are found to operate at similar price levels over the studied period. This leads to distinguish pairs of supermarket competitors, on the one hand, and pairs of oil company and independent gas stations on the other hand.

\begin{figure}[H]
\centering
\caption{Empirical distribution functions of rank reversals (raw prices)}
\label{tab:ecdf_rr_distance}
\begin{subfigure}{.49\textwidth}
\centering
\includegraphics[width=\textwidth]{graphs/ecdf_rank_reversals_le_2c.png}
\caption[short]{Pair average price spread $\le$ 2 cent/litre}
\end{subfigure}
\begin{subfigure}{.49\textwidth}
\centering
\includegraphics[width=\textwidth]{graphs/ecdf_rank_reversals_le_1c.png}
\caption[short]{Pair average price spread $\le$ 1 cent/litre}
\end{subfigure}
\end{figure}

Figure~\ref{fig:pct_reversed_pairs} displays the percentage of pairs of gas stations whose price rank is reversed on each day of the period studied. Among pairs of gas stations built with a maximum distance of 3km, the percentage of reversed pairs fluctuates between 5.4\% and 15.3\%  (mean 8.1\%). The "No differentiation" series results from a focus on pairs which exhibit an average price difference below 2c/l. Among these pairs, the minimum percentage of reversed pairs fluctuates between 13.8\% and 29.3\% (mean 19.5\%). From a consumer viewpoint, this translates in one chance in five to pay the highest price upon patronizing among two competitors of similar price level the one which is cheaper most of the time.

\begin{table}[H]
\renewcommand{\arraystretch}{0.8} % too large by default...
\centering
\caption{Regressions of pair price dispersion}
\begin{tabular}{ll.....} %{lrrrrrr}
\hline
\hline
                  & Dependent  &             & \multicolumn{4}{c}{Quantile regressions} \\
                  & Variable   & \multicolumn{1}{c}{OLS}   & \multicolumn{1}{c}{25\%}  & \multicolumn{1}{c}{50\%}  & \multicolumn{1}{c}{75\%}  & \multicolumn{1}{c}{90\%} \\
\hline
All pairs         & $r_{ij}$   & -0.035^{**} & -0.030^{**} & -0.047^{**} & -0.043^{**} & -0.033^{**} \\
(N = 3,345)       &            & [6.57]      & [4.01]      & [5.12]      & [5.34]      & [4.45]      \\
                  & $std_{ij}$ & -0.001^{**} & -0.001^{**} & -0.001^{**} & -0.001^{**} & -0.002^{**} \\
                  &            & [4.73]      & [3.09]      & [3.53]      & [3.52]      & [2.75]      \\
Supermarkets      & $r_{ij}$   & -0.026^{**} & -0.016^{*}  & -0.027^{**} & -0.043^{**} & -0.036^{*}  \\
(N = 1,488)       &            & [3.73]      & [2.56]      & [2.77]      & [2.78]      & [2.34]      \\
                  & $std_{ij}$ & -0.000^{}   & -0.000^{}   & -0.000^{}   & 0.000^{}    & -0.000^{}   \\
                  &            & [1.01]      & [0.71]      & [1.04]      & [0.25]      & [0.54]      \\
Oil \& Ind        & $r_{ij}$   & -0.033^{*}  & -0.057^{**} & -0.039^{*}  & -0.02^{}    & 0.017^{}    \\
(N = 277)         &            & [2.60]      & [2.73]      & [2.17]      & [1.29]      & [1.16]      \\
                  & $std_{ij}$ & -0.003^{**} & -0.003^{**} & -0.003^{**} & -0.001^{}   & -0.002^{}   \\
                  &            & [3.48]      & [3.85]      & [3.76]      & [1.04]      & [1.02]      \\
\hline
\hline
\end{tabular}%
\label{tab:regs_pairs}
\end{table}

A clear ranking of empirical distribution functions of rank reversals can be observed among pairs of gas stations depending on distances (Figure~\ref{tab:ecdf_rr_distance}). This is consistent with the idea that nearby gas stations compete in a (virtually) complete information setting, in which there is no reason to expect rank reversals. Conversely, distance can create an information issue for other pairs, preventing the existence of an equilibrium in pure strategy. A formal test, proposed by \cite{TAP11}, consists in regressing measures of price dispersion on a dummy variable which identifies competitor pairs for which a low separating distance implies low search costs. Denoting $r_{ij}$ the rank reversals between gas stations $i$ and $j$, $\mathbbm{1}{\text{Corner}_{ij}}$ an indicator for whether the stations are at the same corner, and $X_{ij}$ other control variables, the regression writes:

\begin{equation}
\text{P}_{it}= \beta_0 + \beta_1 \mathbbm{1}{\text{Corner}_{ij}} + X_{ij} + \epsilon_{ij}
\end{equation}

Results (Table~\ref{tab:regs_pairs}) support the hypothesis that limited consumer information is linked to price dispersion. Among supermarkets, competitors are often found to set the very same price, which, given the relatively low prices and absence of evidence of collusion, suggests fierce price competition. Daily price records allow to uncover a number of situations in which one gas station appears to act as a leader.

\section{Market price dispersion}

The following section extends the analysis to markets, investigating how variations in cost and competition intensity impact market price dispersion. The first approach consists in adapting the method employed by \cite{TAP11} to obtain comparable results. The richness of the data is then used to evaluate how gas station price distributions are affected by the intensity of competition.

The first approach in terms of market definition consists in considering each gas station successively as the center of a market delimited by a circle of a given radius. Price dispersion is then measured on each day as the range and the empirical standard deviation in prices. This allows to observe how price dispersion varies with cost variations over time (similar across all markets) and the intensity of competition (number of competitors within each market). Results on the latter must yet be analyzed with caution. First, it is clear that the number of gas stations within an area reflects demand and thus may not provide an accurate measure of the competitiveness of the market. Second, the empirical range and standard deviation are not unbiased estimators of the true range and standard deviation. Their downward bias is all the more pronounced as the size of the sample is small, such as in the gas of gas station local markets. In order to address these issues, dispersion is measured directly for each gas station, and regressed on its number of competitors.

As noted by \cite{TAP11}, considering each gas station as the center of a market leads to attribute a lot of weight to markets which have a high gas station density. A simple algorithm is used to obtain non overlapping markets.  For each gas station, the set of competitors (within 3 or 5 km) is compared to set of competitors from the competitors. If the latter is found to be included in the former, a market is identified. The final list of non overlapping market only includes markets which are obtained both with a 3 and 5 km distance.

Two treatments of differentiation are contemplated. In a first approach, all gas stations are considered to compete in the same market. Price dispersion is computed with price residuals in order to account for gas station specific characteristics. A scenario involving market segmentation is then investigated. Oil and independent gas stations compete in a "high price" market, while supermarket and discounter gas stations form a "low price" market. The analysis is then run both with raw prices and price residuals.

Table~\ref{tab:stats_des_markets} reports descriptive statistics of price dispersion at the market level. All measures of price dispersion can be seen to drop significantly when residual prices are used and under the market segmentation scenario. For instance, under the simple 3 km radius market definition, gains from search, are estimated to be 1.25 euro cents per liter with residual prices while they were 3.93 euro cents per liter with raw prices. Under segmentation, dispersion is higher within "high price" gas station markets (2.90 vs. 0.97 euro cents per liter gains from search with raw prices). The variations observed between dispersion measures computed with raw and residual prices confirm than "high price" gas stations tend to be more differentiated than "low price" gas stations.

The results of regressions of measured price dispersion on the number of firms and cost are reported in Table~\ref{tab:regs_markets}. Formally, with $PriceDispersion_{tj}$ a mesure of price dispersion on market $j$ at date $t$, $MC_t$ a measure of the marginal cost (wholesale diesel) on date $t$, $N_j$ the number of gas stations on market $j$, and $\mathbbm{1}{\text{HighPrice}_{j}}$ an indicator for whether the market is composed by high price gas stations (Oil and independent as opposed to Supermarkets and discounters), the regression writes:

\begin{equation}
\text{PriceDispersion}_{jt}= \beta_0 + \beta_1 MC_t + \beta_2 N_j + \mathbbm{1}{\text{HighPrice}_{j}} + \epsilon_{jt}
\end{equation}

Considering the government intervention between August 2012 and January 2013, estimation outputs are provided for the period starting February 1, 2013 (628 days) and their robustness is checked over the period ending on July 1, 2012 (302 days). They are consistent with results obtained by \cite{TAP11}: dispersion is found to increase with the number of firms and decrease with cost (cf. Table~\ref{tab:stats_des_markets}).

\begin{table}[H]
%\renewcommand{\arraystretch}{0.8} % too large by default...
\centering
\caption{Regressions of market dispersion}
\begin{tabular}{l......} %{lrrrrrr}
\hline
\hline
Markets          & \multicolumn{4}{c}{3 km radius} & \multicolumn{2}{c}{No-overlap} \\
Prices           & \multicolumn{2}{c}{Raw prices} & \multicolumn{2}{c}{Residuals} & \multicolumn{2}{c}{Residuals} \\
Dispersion stat. & \multicolumn{1}{c}{Range} & \multicolumn{1}{c}{Std}   & \multicolumn{1}{c}{Range} & \multicolumn{1}{c}{Std}   & \multicolumn{1}{c}{Range} & \multicolumn{1}{c}{Std} \\
\hline
Cost        & -0.031^{**} & -0.014^{**} & -0.038^{**} & -0.015^{**} & -0.042^{**} & -0.016^{**}            \\
            & [3.86]      & [4.76]      & [4.95]      & [5.04]      & [4.73]      & [4.68]                 \\
Nb firms    & 0.560^{**}  & 0.093^{**}  & 0.265^{**}  & 0.046^{**}  & 0.208^{**}  & 0.033^{**}             \\
            & [15.41]     & [9.40]      & [26.19]     & [15.75]     & [13.56]     & [7.08]                 \\
High price  & 2.703^{**}  & 1.109^{**}  & 0.619^{**}  & 0.258^{**}  &             &                        \\
            & [26.53]     & [26.78]     & [17.87]     & [18.63]     &             &                        \\
Constant    & 1.854^{**}  & 1.386^{**}  & 2.670^{**}  & 1.294^{**}  & 3.370^{**}  & 1.535^{**}             \\
            & [3.71]      & [7.56]      & [6.04]      & [7.48]      & [6.49]      & [7.44]                 \\
\hline
R2          & 0.42  & 0.31  & 0.26  & 0.12  & 0.09  & 0.02                                                 \\
N           & \multicolumn{2}{c}{1,074,894} & \multicolumn{2}{c}{1,074,894} & \multicolumn{2}{c}{261,687}  \\
\hline
\hline
\end{tabular}%
\label{tab:regs_markets}
\end{table}

Given the richness of price data, price dispersion can also be measured at the gas station level, using each residual price distribution. Measures of price dispersion thereby obtained are reported in Table~\ref{tab:station_price_support_stats_des}. Price distributions of low price gas station tend exhibit smaller standard deviations but fatter tails than these of high price gas stations. They are often left-skewed which reflects the use of occasional promotions, often implemented at the chain level. Conversely, no systematic skew is observed for high price gas stations.

\begin{table}[!h]
%\renewcommand{\arraystretch}{0.8} % too large by default...
\caption{Regressions of gas station price residual distributions}
\label{tab:station_price_support_regs}
\centering
\begin{tabular}{l....} %{lrrrrrr}
\hline
\hline
      & \multicolumn{1}{c}{Tr. range} & \multicolumn{1}{c}{Std} & \multicolumn{1}{c}{Tr. range} & \multicolumn{1}{c}{Std} \\
\hline
Constant        & 4.802^{**} & 1.243^{**}  & 4.266         & 1.106          \\
                & [35.45]    & [33.58]     & [230.88]^{**} & [307.05]^{**}  \\
Low type        & -0.507^{**}& -0.138^{**} &               &                \\
                & [4.81]     & [5.31]      &               &                \\
Nb competitors  & 0.021^{**} & 0.008^{**}  & 0.040^{**}    & 0.012^{**}     \\
                & [2.64]     & [3.79]      & [2.75]        & [4.19]         \\
Nb competiors * & -0.039^{*} & -0.014^{**} & -0.064^{**}   & -0.020^{**}    \\
Low type        & [2.13]     & [3.08]      & [3.32]        & [4.90]         \\
\hline
Controls & \multicolumn{1}{c}{} & \multicolumn{1}{c}{} & \multicolumn{2}{c}{Region*type} \\
\hline
R2    & \multicolumn{1}{c}{0.03} & \multicolumn{1}{c}{0.04} & \multicolumn{1}{c}{0.07} & \multicolumn{1}{c}{0.09} \\
N     & \multicolumn{1}{c}{8,647} & \multicolumn{1}{c}{8,647} & \multicolumn{1}{c}{8,647} & \multicolumn{1}{c}{8,647} \\
\hline
\hline
\end{tabular}%
\end{table}

Regression results in Table~\ref{tab:station_price_support_regs} confirm the fact that dispersion is positively associated with the number of competitors for high price gas stations. It is yet not true regarding low price gas stations.


\section{Conclusion}

This paper expands the methodology of \cite{TAP11} to measure and analyse price dispersion in the French retail gasoline market, taking into account the presence of significant differentiation and potential segmentation. Rank reversals are generally found to be less frequent for pairs gas stations separated by a short distance i.e. competitors whose prices are easy to compare for consumers. This result supports the hypothesis of a connection between consumer information and price dispersion. Pairs of competitors which operate at low prices exhibit less rank reversals than those which sell at higher prices, and many are actually observed to set the very same price on a regular basis. Conversely, among high price gas stations, all pairs tend to be characterized by significant static and/or dynamic price dispersion.

At the market level, price dispersion is found to increase with the number of competitors, and decrease with cost. This is consistent with predictions of a model a la \cite{VAR80}. In the presence of significant differentiation, working with raw prices typically leads to largely overestimate price dispersion related to search costs.


\newpage
\bibliography{references}

\newpage

\appendix

\section{Descriptive statistics}

\begin{table}[H]
\begin{threeparttable}
\renewcommand{\arraystretch}{0.8} % too large by default...
\caption{Overview of retailer chains (last day of the period)}
\label{tab:station_chains}
    \begin{tabular}{llrrrrr}
    \hline
		\hline
          &       & \multicolumn{5}{c}{Prices (euro cents) on 2014/12/04} \\
    Type  & Chain & Count & Mean  & Std   & Q75/Q25 & Q90/Q10 \\
		\hline
    \textbf{Supermarkets} &       & \textbf{4,967} &       &       &       &  \\
    Large & Carrefour & 201   & 1.15  & 0.03  & 1.02  & 1.05 \\
    Large & Auchan & 118   & 1.16  & 0.03  & 1.02  & 1.05 \\
    Large & Cora  & 111   & 1.18  & 0.04  & 1.04  & 1.08 \\
    Large & Geant Casino & 97    & 1.16  & 0.02  & 1.03  & 1.04 \\
    Large and medium & Intermarche & 1,391 & 1.17  & 0.03  & 1.03  & 1.06 \\
    Large and medium & Systeme U & 771   & 1.16  & 0.03  & 1.03  & 1.06 \\
    Large and medium & Leclerc & 585   & 1.15  & 0.02  & 1.03  & 1.06 \\
    Small & Carrefour market & 717   & 1.18  & 0.03  & 1.03  & 1.05 \\
    Small & Carrefour contact & 233   & 1.20  & 0.03  & 1.03  & 1.05 \\
    Small & Simply (Auchan) & 222   & 1.20  & 0.03  & 1.04  & 1.07 \\
    Small & Casino & 200   & 1.21  & 0.03  & 1.03  & 1.06 \\
    Small & Intermarche contact & 112   & 1.20  & 0.03  & 1.04  & 1.07 \\
    Other supermarkets &       & 209   &       &       &       &  \\
		\hline
    \textbf{Oil and independent} &       & \textbf{3,770} &       &       &       &  \\
    Oil   & Total & 1,283 & 1.27  & 0.03  & 1.03  & 1.07 \\
    Oil   & Elan (Total) & 233   & 1.32  & 0.04  & 1.05  & 1.08 \\
    Oil   & Agip  & 116   & 1.25  & 0.03  & 1.02  & 1.06 \\
    Oil (Independent) & BP    & 275   & 1.26  & 0.04  & 1.03  & 1.07 \\
    Oil (Independent) & Esso  & 147   & 1.27  & 0.05  & 1.05  & 1.11 \\
    Independent & Avia  & 376   & 1.27  & 0.05  & 1.03  & 1.07 \\
    Independent & Dyneff & 55    & 1.26  & 0.04  & 1.05  & 1.07 \\
    Oil - discount & Total access & 616   & 1.16  & 0.03  & 1.02  & 1.04 \\
    Oil (Independent) - discount & Esso express & 318   & 1.16  & 0.02  & 1.02  & 1.04 \\
    Other independent &       & 351   &       &       &       &  \\
		\hline
    \textbf{Total} &       & \textbf{8,737} &       &       &       &  \\
    \hline
		\hline
\end{tabular}
\begin{tablenotes}
			\small
			\item Sub-classification of type for supermarkets is meant to reflect what consumers can infer from chain name (as provided on the price comparison website).
      \item Gas stations are considered independent when they are neither operated by a supermarket nor part of a chain operated by an oil company. BP and Esso (including Esso Express) gas stations  have an intermediary status: they have been sold to third-party companies with a supply agreement and the right to exploit the brand name.
\end{tablenotes}
\end{threeparttable}
\end{table}

\newpage

\begin{table}[H]
\begin{threeparttable}
\renewcommand{\arraystretch}{0.8} % too large by default...
\caption{Station-level summary statistics}
\label{tab:station_stats_des}
\begin{tabular}{lr|rr|rr}
    \hline
    \hline
		{}          & All   & Super-   & Others & Paris  & Others \\
    {}          &       & markets  &        & region & \\
    \hline
    Nb stations &       &       &       &       &  \\
    All periods & 9,217 & 5,055  & 4,162  & 918   & 8,299 \\
    Nb daily observations (avg) & 7,896  & 4,400  & 3,496  & 776   & 7,119 \\
		\hline
    Competition &       &       &       &       &  \\
    Distance (km) to closest rival & 2.3  & 2.6  & 1.8  & 1.2  & 2.4 \\
    Nb of rivals within 3 km       & 3.2  & 2.5  & 4.1  & 6.9  & 2.8 \\
    Nb of rivals within 5 km       & 6.6  & 4.9  & 8.6  & 17.8 & 5.3 \\
    \hline
    Price and Markup (euro per liter) &       &       &       &       &  \\
    Price after tax            & 1.37  & 1.33  & 1.4   & 1.38  & 1.36 \\
    Price excl. Tax            & 0.71  & 0.68  & 0.74  & 0.72  & 0.70 \\
    Markup over wholesale cost & 0.10  & 0.08  & 0.13  & 0.11  & 0.10 \\
    \hline
    Retail price variations (euro cents per liter) &       &       &       &       &  \\
    Daily price change probability & 0.17  & 0.18  & 0.15  & 0.21  & 0.16 \\
    Average price increase         & 1.30   & 1.29  & 1.31  & 1.05  & 1.33 \\
    Average price decrease         & 1.47   & 1.48  & 1.46  & 1.18  & 1.50 \\
    \hline
    \hline
\end{tabular}
\begin{tablenotes}
			\small
      \item Price statistics are obtained by i) computing the average for each station over time ii) taking the average over stations. Costs of transportation and distribution are to be subtracted from the markup, leading to a net margin generally estimated at c. 1 euro cent per liter.
\end{tablenotes}
\end{threeparttable}
\end{table}

\begin{figure}[H]
\centering
\caption{Histograms of station price changes}
\begin{subfigure}{.49\textwidth}
\centering
\includegraphics[width=\textwidth]{graphs/hist_station_price_chge_proba.png}
\caption[short]{Station daily price change probabilities}
\label{fig:hist_station_price_chge_proba}
\end{subfigure}
\begin{subfigure}{.49\textwidth}
\centering
\includegraphics[width=\textwidth]{graphs/hist_station_mean_price_chge.png}
\caption[short]{Station average price variations}
\label{fig:hist_station_mean_price_chge}
\end{subfigure}
\end{figure}

\section{Rank reversals}

\begin{table}[H]
\begin{threeparttable}
\renewcommand{\arraystretch}{0.8} % too large by default...
\caption{Pair rank reversals}
\label{tab:stats_pair_rank_reversals}
\begin{tabular}{lrrrrr}
\hline
\hline
      & Nb.   & \multicolumn{2}{c}{Price spread} & \% Rank   & \% Same\\
\cmidrule(l{2pt}r{2pt}){3-4}
      & pairs & Abs. avg.  & Std. dev.           & reversals & price \\
\hline
\textbf{$d_{ij} \le  3 km$} &       &       &       &       &  \\
\hline
\multicolumn{2}{l}{\textbf{No price spread restriction}}  &       &       &       &  \\
All types & 12,298 & 5.1   & 1.6   & 8.3   & 7.7 \\
Supermarket pairs & 2,331 & 1.0   & 1.3   & 13.7  & 26.6 \\
Oil/independent pairs & 1,126 & 2.7   & 1.9   & 14.9  & 2.9 \\
Discounter pairs & 200   & 0.4   & 1.1   & 28.3  & 13.0 \\
Discounter vs. supermarket pairs & 1,484 & 0.9   & 1.3   & 18.9  & 12.5 \\
\hline
\multicolumn{2}{l}{\textbf{Price spread $\le$ 1 cent per liter}}   &       &       &       &  \\
All types & 3,429 & 0.4   & 1.2   & 22.4  & 24.5 \\
Supermarket pairs & 1,524 & 0.4   & 1.1   & 16.9  & 37.9 \\
Oil and independent pairs & 286   & 0.5   & 1.8   & 35.3  & 5.8 \\
Discounter pairs & 176   & 0.3   & 1.0   & 30.0  & 13.9 \\
Discounter vs. supermarket pairs & 1,034 & 0.5   & 1.2   & 22.9  & 16.0 \\
\hline
\textbf{$d_{ij} \le  5 km$} &       &       &       &       &  \\
\hline
\multicolumn{2}{l}{\textbf{No price spread restriction}}   &       &       &       &  \\
All types & 25,076 & 5.2   & 1.7   & 8.5   & 6.0 \\
Supermarket pairs & 4,515 & 1.1   & 1.4   & 14.6  & 21.3 \\
Oil/independent pairs & 2,680 & 3.0   & 2.0   & 13.3  & 2.3 \\
Discounter pairs & 401   & 0.5   & 1.2   & 28.4  & 10.4 \\
Discounter vs. supermarket pairs & 3,067 & 1.0   & 1.4   & 19.9  & 10.3 \\
\hline
\multicolumn{2}{l}{\textbf{Price spread $\le$ 1 cent per liter}} &       &       &       &  \\
All types & 6,413 & 0.4   & 1.3   & 24.2  & 20.4 \\
Supermarket pairs & 2,758 & 0.4   & 1.2   & 18.9  & 32.1 \\
Oil and independent pairs & 575   & 0.5   & 1.9   & 35.5  & 5.1 \\
Discounter pairs & 330   & 0.3   & 1.1   & 31.5  & 11.5 \\
Discounter vs. supermarket pairs & 2,003 & 0.5   & 1.2   & 25.0  & 13.6 \\
\hline
\hline
\end{tabular}
\begin{tablenotes}
			\small
      \item Except for the first column, all figures are averages of statistics computed at the pair level. As regards the price spread, the absolute value of the average (Abs. avg.) accounts for the existence of persistent price differences (0 if both gas stations are equally expensive over the long term), and the standard deviation (Std. dev.) measures dispersion around the long term price difference (0 if both gas stations always set the same price).
\end{tablenotes}
\end{threeparttable}
\end{table}

\begin{figure}[H]
    \caption{Percentage of rank reversals among pairs}
    \label{fig:pct_reversed_pairs}
	\centering
		\includegraphics[width=\textwidth]{graphs/macro_rank_reversals_le_1c.png}
    \floatfoot{Series represent for each day the percentage of pairs observed where the usual price order is not respected (reversed rank). No differentiation implies that pairs exhibit an average price difference below 1c/l.}
\end{figure}

\section{Market price dispersion}

\begin{table}[H]
  \caption{Market-level summary statistics}
\begin{tabular}{lrrrrrrrr}
\hline
\hline
      & \multicolumn{2}{c}{3km}    & \multicolumn{2}{c}{3km}       & \multicolumn{2}{c}{3km}        & \multicolumn{2}{c}{Non-} \\
      & \multicolumn{2}{c}{all}    & \multicolumn{2}{c}{low price} & \multicolumn{2}{c}{high price} & \multicolumn{2}{c}{overlapping} \\
\hline
Nb sellers & 5.99  & (3.36) & 3.81  & (1.13) & 5.66  & (3.41) & 4.36  & (2.06) \\
Nb sellers observed & 5.74  & (3.22) & 3.70  & (1.13) & 5.32  & (3.20) & 4.18  & (2.04) \\
\hline
\multicolumn{3}{l}{Raw prices (euro cents per liter)}          &       &       &       &       &       &  \\
Range & 9.70  & (4.54) & 2.24  & (1.97) & 6.04  & (3.82) & 8.61  & (4.61) \\
Standard deviation & 3.80  & (1.71) & 0.92  & (0.80) & 2.21  & (1.23) & 3.65  & (1.96) \\
Gain from search & 3.93  & (2.27) & 0.97  & (0.88) & 2.90  & (1.74) & 3.39  & (2.07) \\
\hline
\multicolumn{3}{l}{Residual prices (euro cents per liter)}      &       &       &       &       &       &  \\
Range & 2.50  & (1.75) & 1.60  & (1.37) & 2.91  & (1.86) & 2.01  & (1.55) \\
Standard deviation & 0.88  & (0.57) & 0.64  & (0.55) & 1.01  & (0.61) & 0.79  & (0.59) \\
Gain from search & 1.25  & (0.95) & 0.80  & (0.72) & 1.39  & (1.01) & 1.01  & (0.84) \\
\hline
Nb markets & \multicolumn{2}{c}{5,501} & \multicolumn{2}{c}{1,604} & \multicolumn{2}{c}{883} & \multicolumn{2}{c}{604} \\
Nb obs     & \multicolumn{2}{c}{5,852,850} & \multicolumn{2}{c}{1,709,055} & \multicolumn{2}{c}{918,581} & \multicolumn{2}{c}{641,314} \\
\hline
\hline
\end{tabular}
\label{tab:stats_des_markets}
\end{table}

\section{Gas station price distributions}

\begin{table}[H]
  \caption{Gas station residual price distributions}
	\label{tab:station_price_support_stats_des}
\begin{tabular}{rrrrr}
\hline
\hline
      & \multicolumn{2}{c}{Low price} & \multicolumn{2}{c}{High price} \\
\hline
Nb stations & \multicolumn{2}{c}{5,603} & \multicolumn{2}{c}{3,044} \\
Std   & 1.10  & (0.33) & 1.26  & (0.47) \\
Kurtosis & 2.51  & (3.92) & 0.80  & (2.30) \\
Skewness & -0.38 & (0.91) & 0.13  & (0.66) \\
Range & 7.52  & (2.16) & 7.12  & (2.25) \\
Trimmed range 5\% & 4.27  & (1.36) & 4.83  & (1.80) \\
Trimmed range 10\% & 3.42  & (1.11) & 4.02  & (1.54) \\
\hline
\hline
\end{tabular}
\end{table}

\section{Sources of rank reversals}

Data provide several examples of high rank reversal statistics unrelated to consumer lack of information. The conversion by Total S.A. of about c. 300 Total gas stations to its discount brand, Total Access, generates many observations with meaninglessly high rank reversals. Indeed, the conversion is generally accompanied by a decrease in prices of c. 10 euro cents per liter, which has been found to trigger very moderate adjustments by competitors (\cite{CHA16}). A converted Total gas station can thus often be observed to be consistently more expensive than its competitor(s) at the beginning of the period, and then cheaper once the conversion has occurred. Figure ~\ref{fig:rr_total_access} provides an example for which rank is measured to be reversed 47\% of the time.

\begin{figure}[H]
    \caption{Daily Brent and French diesel retail prices}
	\centering
		\includegraphics[width=\textwidth]{graphs/example_spurious_rr_total_access.png}
\label{fig:rr_total_access}
\end{figure}

More generally, such an issue may arise whenever a shock affects a gas station differently from the way it affects its closest competitors, so that relative competitive advantages change over time. In order to reduce the potential influence of this issue, gas stations which are found to implement a significant change in pricing policy over the period are excluded, and various constraints are imposed on the maximum length of rank reversals or their number to check the robustness of results.

Descriptive statistics performed at the gas station level have also shown that a few gas stations appeared to implement dynamic price discrimination. In order to detect such practices, the number and regularity of successive inverse price changes were measured for each gas station. The most commonly observed pattern is a surge in prices during weekends. Such a pricing policy, whenever it is implemented unilaterally by a gas station in one market, can generate rank reversals which are unrelated to the use of mixed strategies. Figure~\ref{fig:rr_dynamic_price_discrimination} provides an illustration of this phenomenon. Since only a handful of gas stations are concerned, dynamic price discrimination does not have a significant impact on the present analysis.

\begin{figure}[H]
    \caption{Daily Brent and French diesel retail prices}
	\centering
		\includegraphics[width=\textwidth]{graphs/example_spurious_rr_dynamic_price_discrimination.png}
\label{fig:rr_dynamic_price_discrimination}
\end{figure}

Finally, several papers have documented the existence of Edgeworth cycles on retail gasoline markets. The relatively low frequency of price changes, and the symmetry in positive and negative price variation distributions suggest that such cycles are unlikely to play a significant role on the French retail gasoline market.

\end{document} 