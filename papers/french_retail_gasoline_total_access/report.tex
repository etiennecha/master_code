\documentclass[11pt]{article}
\usepackage[T1]{fontenc}
\usepackage[latin1]{inputenc}
\usepackage{enumerate}
\usepackage{setspace}
\usepackage{amsmath,amssymb,amsthm}
\usepackage{graphicx}
\usepackage{bbm}
\usepackage[round]{natbib}
\usepackage[nohead]{geometry}
\usepackage[bottom]{footmisc}
\usepackage{indentfirst}
\usepackage{endnotes}
\usepackage{graphicx}%
\usepackage{eurosym}
\usepackage{array}
\usepackage{booktabs}
\usepackage{caption}
\usepackage{subcaption}
% \usepackage[hidelinks]{hyperref}
\usepackage{floatrow} %[capposition=top]
\floatsetup{footposition=bottom,capposition=top}
\renewcommand{\labelitemi}{--}
\renewcommand{\labelitemii}{$\bullet$}
\bibliographystyle{chicago}
% \geometry{left=1in,right=1in,top=1.00in,bottom=1.0in}
\let\olditemize\itemize
\renewcommand{\itemize}{
  \olditemize
  \setlength{\itemsep}{-1pt}
}

\begin{document}

\title{Creation of a low cost branch by a major retailer: Evidence from the French gasoline market\ \\ \ \\(Very preliminary)}
\author{Etienne Chamayou\thanks{e-mail:
\textit{etienne.chamayou@ensae.fr}}\medskip\\{\normalsize CREST and Department of Economics, Ecole Polytechnique }}
\maketitle

\sloppy%

\onehalfspacing

\textbf{Abstract:}

Total S.A is one of the five "supermajor" oil companies in the world and operates the largest gas station network in France. End of 2011, the company launched a new brand, "Total Access", with the stated goal of recapturing market shares lost due to the development of supermarket gas station chains. Two years later, the new network had reached the size of c. 600 gas stations, exlusively through the rebranding of existing stations. The creation of the brand and its impact on competition is studied using daily price data obtained from a comparison website. For half of the rebranded stations, the rebranding is accompanied by a sharp decline in price. At an aggregate level, competitors' prices are found not to exhibit any significant adjustment. [TODO: DESC RESULTS]

\strut

\textbf{Keywords:} Competition, Gasoline

\strut

\textbf{JEL Classification Numbers:} L13

\pagebreak%
%\doublespacing

\section{Introduction}

While supermarket chains only accounted for 12\% of retail gasoline sales in 1980, low prices have allowed them to reach a current 60 \% market share. Over last years, several traditional gas station operators have exited the markets (e.g. Shell, BP) or engaged in large divestment operations (e.g. Esso). On the other end, the largest operator, Total, decided end of 2011 to launch a new chain regrouping existing stations capable of coping with the price competition imposed by supermarkets. Within the next two years, about 600 hundred existing stations were thus rebranded to integrate the "Total Access" network. While half of them were previously operated under the chain "Elf" (inheritance from a merger) and already posted low prices, rebranding of the other c. 300 "Total" stations was accompanied by a sharp drop in prices.
Concerned about consumer information regarding gas station prices, the French governement has launched a comparison website in 2006 on which it is mandatory for gas stations selling above $500m^3$ per year to keep prices posted. The creation of "Total Access" and the availability of rich price dataset on the period thus offer an interesting opportunity to assess the competitiveness of the French retail gasoline market.
[TODO: DESC RESULTS]

\section{The French retail gasoline market}

The size of the French gas station network has been decreasing at a steady pace over the last decades, from c. 40,000 in 1980 to c. 12,000. Industry reports attribute this decline to two main causes: technological improvements affecting fuel consumption and the development of large supermarket gas stations with low price policies.

End of 2013, according to Nielsen, there were 11,476 gas stations in France, of which 4,979 were operated supermarkets. About 1,500 gas stations were reported to sell less than 500m$^{3}$, with the median gas station selling between 1,000 and 3,000m$^{3}$.

No chain implements a national or regional unique price policy. According to industry reports and interviews, only a small share of gas stations belonging to large retail chains decide prices. In general, chains have a team dedicated to pricing which uses information communicated by gas station managers about local competition to optimize prices.

Key cost components are the cost of wholesale gasoline, including delivery fees,  gas station operating expenses, and taxes. Taxes included a fix part called TICPE, which slightly varies across regions, and the classical Value-Added Tax (19.6\% over the period studied, which bear on cost and TICPE).

Regarding consumers, it is worth noticing that companies are offered card programs which allow them to monitor employees' consumptions and obtain rebates. An important implication is that the price actually paid may then not be the price posted at the gas station where fuel is obtained.

Consumers can get information about prices from a variety of sources including gps, mobile phone applications (e.g. Zagaz, Carbeo, Essence Free) or compure and mobile phone browser (Prix-Carburants.gouv.fr).

\section{Data}

\subsection{Data overview}

Data about daily diesel prices, gas station locations, opening hours and amenities were obtained from the governemental price  website prix-carburants.gouv.fr. The "Total Access" rebranding was found to be often declared with significant delay on the website as compared to a document from Total providing (non exhaustive) information about scheduled station renovations. While all ex "Total" gas stations could be fixed since the rebranding is accompanied by a sharp drop in prices, some uncertainty remains for ex "Elf" gas stations.

The period covered in the data starts on September 4, 2011 and ends on December 4, 2014. On some periods, prices could not be collected for various technical reasons, so that the maximum number of price observation for a gas station is 1,074. The database contains 10,231 gas stations, which is consistant with the number of gas station in France and the fact that small gas stations do not have to keep prices posted on the website).

\subsection{"Total Access"}

At the end of the period studied, there were 641 Total Access registered in the database, of which 374 were previously operated under the brand "Total" and 250 as "Elf" (others were either operated within another chain or information is missing). 

For each rebranded station, a change in pricing policy was investigated by looking for the date which maximized the difference ( before and after the date) in estimated station fixed effect vs. the average national price. While no significant discontinuity could be identified for "Elf-Total Access" gas stations, it was the case for 344 "Total-Total Access" gas stations.

\subsection{Competitors}

In the following analysis, two stations are considered to be competitors whenever they are separated by a distance of less than 3 km (as the crow flies). In the analysis of the rebranding from "Total" to "Total Access", all stations which do not belong to the group Total and are located within 3 km of a Total-Total Access are considered as competitors. In the case of "Elf", on top of these conditions, it is imposed that no Total-Total Access is located within a radius of 10km. Indeed, with Elf-Total Access, the goal is to observe the impact of the rebranding without a price drop. Robustness checks were performed to ensure than findings were not sensitive to the radius used.

\section{Results}

\subsection{Aggregate fixed-effects estimation}

The richness of data allows to perform a regression with station-level fixed effects as well as day fixed-effects. The presence of station-level fixed effects is necessary given the differentiation observed between stations (location, brand image etc.), while day fixed-effects are included to account for variations in market conditions common to all gas stations. Since the rebranding occurs progressively over a period of two years, the regressions can be run by including stations of interest only (i.e. control is ensured by stations not yet affected by a rebranding and/or change in pricing policy).

\begin{align*}
p_{it} = \mu + \alpha_i + \delta_t + \theta r_{it} + \epsilon_{it}
\end{align*}
with $\mu$ a constant, $\alpha_i$ the station fixed-effects, $\delta_t$ the day fixed-effects, $r_{it}$ indicator equal to 1 in case of Rebranding. Table \ref{table:ta_fe_agg} reports the results of six regressions run according to this specification. In column (1), only the price of ex-Elf stations rebranded Total Access are included so that the Rebranding indicator captures the change in pricing policy implemented by Total. The same goes for column (4) with ex-Total stations rebranded Total. As expected, there is no significant change for ex-Elf stations (-0.1 cent), while the sharp drop in price for ex "Total" stations is appropriately captured with a nearly -10 cent estimated effect.

In columns (2) and (3), regressions are run with prices of competitors of Elf-Total Access gas stations. In column (3), the treatment is interacted with a categorical variable accounting for the "type" of the competitor: supermarket gas station, oil company gas station, or independent.

 The rebranding of Elf to Total Access....

 A similar analysis is reported in columns (5) and (6) for competitors of Total-Total Access gas stations...


\begin{table}[h]
\centering
\def\sym#1{\ifmmode^{#1}\else\(^{#1}\)\fi}
\caption{Aggregate FE estimates}
\begin{tabular}{lcccccc}
\toprule
\toprule
\multicolumn{7}{c}{Dependent variable:}\\
\multicolumn{7}{c}{Retail price for diesel (cents)}\\
\hline
{}                      & (1)                            & (2)                        & (3)                           & (4)                              & (5)                               & (6)                   \\
{}                      & Elf-TA                        & Elf-TA                    & Elf-TA                      & Total-TA                       & Total-TA                      & Total-TA                   \\
{}                      & station                      & competitors          & competitors             & station                        & competitors                & competitors              \\
\hline
Rebranding       & -0.140\sym{*}         & 0.036                    &                                 & -9.826\sym{***}       & -0.345\sym{***}      &                                  \\
                         & (0.067)                     & (0.128)                  &                                & (0.105)                       & (0.062)                      &                                  \\
\hline
Rebr. vs.           &                                  &                               & -0.329\sym{*}      &                                    &                                     & -0.400\sym{***}     \\
Supermarket     &                                  &                               & (0.139)                  &                                    &                                     & (0.060)                    \\
\hline
Rebr. vs.           &                                  &                               & 0.776\sym{**}      &                                    &                                   & -0.053                        \\
Oil company      &                                  &                                & (0.273)                 &                                    &                                    &  (0.125)                      \\
\hline
Rebr. vs.           &                                  &                               & 0.751                     &                                    &                                   & -0.386                       \\
Independent     &                                  &                               & (0.888)                  &                                    &                                    & (0.233)                     \\
\hline
Nb stations       & 250                           & 144                        & 144                         & 313                              & 604                            & 600                            \\
Obs                   & 257,600                    & 146,122                &  146,122                 & 322,273                      &  606,256                    & 439,600                     \\
R-square           & 0.910                        & 0.542                    &  0.581                     & 0.961                          &  0.588                        &  0.530                       \\
\hline\hline
\multicolumn{7}{l}{\footnotesize Standard errors in parentheses}\\
\multicolumn{7}{l}{\footnotesize \sym{*} \(p<0.05\), \sym{**} \(p<0.01\), \sym{***} \(p<0.001\)}\\
\end{tabular}
\floatfoot{All regressions are run with two way fixed effects for day and station. Standard errors are clustered at the station level}\\
\label{table:ta_fe_agg}
\end{table}

\subsection{Station level fixed-effect Estimation}

Results from aggregate analysis and observations suggest that there is significant heterogeneity in the reaction of gas stations. In order to gain a better understanding of this heterogeneity, a similar analysis is performed at the gas station level. For each affected gas station, a regression is run with the prices of a Total Access competitor and prices of gas stations located in the same "départment" (Metropolitan France is composed of 96 such regions) but which are not affected by the rebranding.

\section{Conclusion}

\newpage

\bibliography{references}

\newpage

\appendix

\section{Maps}

\end{document} 