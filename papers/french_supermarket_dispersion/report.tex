\documentclass[english]{article}
\usepackage[T1]{fontenc}
\usepackage[latin1]{inputenc}
\usepackage{enumerate}
\usepackage{setspace}
\usepackage{amsmath,amssymb,amsthm}
\usepackage{graphicx}
\usepackage{bbm}
\usepackage[round]{natbib}
\usepackage[nohead]{geometry}
\usepackage[bottom]{footmisc}
\usepackage{indentfirst}
\usepackage{endnotes}
\usepackage{graphicx}%
\usepackage{eurosym}
\usepackage{array}
\usepackage{booktabs}
\usepackage{caption}
\usepackage{subcaption}
\usepackage{tabularx}
\usepackage[flushleft]{threeparttable}
% \usepackage[hidelinks]{hyperref}
\usepackage{floatrow} %[capposition=top]
\floatsetup{footposition=bottom,capposition=top}
\renewcommand{\labelitemi}{--}
\renewcommand{\labelitemii}{$\bullet$}
\bibliographystyle{chicago}
% \geometry{left=1in,right=1in,top=1.00in,bottom=1.0in}
\let\olditemize\itemize
\renewcommand{\itemize}{
  \olditemize
  \setlength{\itemsep}{-1pt}
}

\begin{document}

\title{Competition between French grocery stores: Evidence from a price comparison website\ \\ \ \\Working paper}
\author{E. Chamayou\thanks{e-mail:
\textit{etienne.chamayou@ensae.fr}} \\ CREST-LEI}
\maketitle

\sloppy%

\onehalfspacing

\textbf{Abstract:}

The French grocery store chain Leclerc operates a price comparison website which allows to compare each of its stores with some local competitors, and performs chain comparisons at the national level. This papers uses price data collected from the comparison website to investigate static and dynamic price dispersion across French grocery stores. Although chains are found to largely determine store pricing policies, chain level comparisons are often of little informative value given heterogeneity observed at the store level. Furthermore, store comparison results tend to be sensitive to available product sets and vary significantly over time. Findings therefore suggest that static and dynamic price dispersion makes accurate price comparisons very costly for consumers.

\strut

\textbf{Keywords:}

\strut

\textbf{JEL Classification Numbers:} D83, L81, M31

\pagebreak%
%\doublespacing
\begin{flushleft}

\end{flushleft}
\section{Introduction}

Since the seminal paper of \cite{STI61}, a large literature has investigated the link between "consumer ignorance" and price dispersion i.e. the persistence over time of different prices for a homogeneous good in a given market. Following \cite{VAR80}, a rich theoretical paradigm has emerged in which price dispersion results from price randomization by sellers in equilibrium. Empirical research, on the other hand, has long been hampered by a scarcity of relevant panel data. For instance, \cite{LAC02} used CPI data of four grocery store products sold in Israel over four years. Dynamic price dispersion was observed in the form of sellers frequently changing quartiles in the price distribution over months at the national level. Data were yet not rich enough to observe price dispersion within local markets. \cite{TAP11} have overcome this limitation with retail gasoline data. Studying pairs of competing gas stations, they observe that the distance separating gas stations, used as a proxy for consumer information, is positively correlated with rank reversals (i.e. changes in the price ranking of the two gas stations over time). They also investigate price dispersion at the market level and argue that their findings are globally consistent with \cite{VAR80}

The present paper adapts the tools used \cite{TAP11} to analyse dispersion in retail grocery store prices and thus elaborate on \cite{LAC02} observations. Using rich price data collected from a comparison website, it documents static and dynamic price dispersion between French retail grocery stores. The first part of the paper provides a description of the data through a review of the methodology used by the website to compare prices. 

\section{Price comparison on Quiestlemoinscher.com}

As of 2015, the French retailing industry was dominated by six firms, which accounted for over 80\% of total sales: Carrefour and Leclerc were the two largest with c. 20\% market shares each, followed by Intermarche (14\%), Casino (12\%), Auchan (11\%) and Systeme U (10\%). An important distinction between firms lies in the ownership structures. While Carrefour, Casino and Auchan own most of the stores operated under their brands, Leclerc, Intermarche and Systeme U are essentially franchise networks.

The creation of the comparison website Quiestlemoinscher (Qlmc) is part of a long term strategy of the group Leclerc to prove the competitiveness of its prices. Soon after the launch in May 2006, Carrefour filed a complaint about the lack of transparency and potential biases in comparisons. The website was forced to close by a court decision. An updated version of the website was released on November 2006 and has since then remained in operation. Legal proceedings nervetheless continued until the rejection by the court of cassation of Carrefour's claims in January 2010.

A major merit of the legal action undertaken by Carrefour is its consequence for the transparency, namely the release of well identified store product price data. The following section provides an overview of the methodology of the comparison website, two crucial aspects of which are the choices of stores and products.

\subsection{Products}

Before 2013, the website only offered comparisons at the chain level. For each chain, according to Qlmc, prices were collected at a sample of store expected to be representative of the store network. Broad constraints were thus imposed on store location and size, while exact store choice was claimed to be random.

From 2013 on, the development of the "drive" concept in France has allowed the comparison website to cover far more stores, and thus to start displaying store level comparisons. The concept of "drive" generally implies that consumers are offered the opportunity to shop online from a physical store (at the same prices) and collect their purchases whenever it suits them. The collection of prices can then be achieved efficiently on the internet, as opposed to traditional store visits. As of March 2015, Qlmc claimed to cover 60\% of the stores of the 10 supermarket chains compared (44\% in August 2013).

Regarding store level comparisons, Qlmc  methodology states that each Leclerc is compared with a selection of its most relevant competitors within 30 km, based on Leclerc managers' expertise. The website also indicates that stores whose surface is smaller than 1,000 $m^2$ are excluded, as well as stores belonging to chains which are deemed to be too differentiated (e.g. discounters). Finally, Leclerc stores are not included among potential competitors.

As of March 2015, only national brand products are covered by the website. Products are identified at the bar code level. There are seven food product categories: meat and fish, vegetables and fruits, bakery, fresh food, frozen food, savoury grocery, sweet grocery, baby food and drinks. Non food products are split in four categories: health and beauty, household, pets and home and textile. Products are further classified within product families. Regarding chain comparisons, the number of products covered within each family is determined by the volume of national hypermarket and supermarket sales, with a global objective of 3,000 products. Within each family, products are chosen based on the national hypermarket and supermarket detention rate. Products whose detention rate is below 30\% (i.e. products referenced by less than 30\% of the stores) are dropped. This led to a total of 2,461 national brand product references covered for March 2015 (2,510 in August 2013). As regards store level comparisons, all products found at both stores are used in comparisons.

\subsection{Price comparison}

The comparison of Leclerc with its competitors follows two simple steps. First, the average price of each product is computed for each chain, provided the product is observed within enough stores of the chain. Leclerc is then successively compared to each of its competitors based on all products for which a chain price was computed. The result displayed on the website is the percentage difference between the price of the basket for the competing chain and for Leclerc:
\begin{align*}
\frac{\sum\limits_{i} P_{iC} - \sum\limits_{i} P_{iL}}{\sum\limits_{i} P_{iL}}
\end{align*}
where $i$ refers to all products in the baskets, $P_{iC}$ and $P_{iL}$ respectively stand for the average price of product $i$ for the competing chain ($C$) and for Leclerc ($L$). The comparison between two stores is very similar except that it uses store prices instead of average chain prices.

\section{Data and replication of price comparisons}

This section provides an overview of the data collected from the website in 2015 and of the replication of its results. Potential biases in terms of store and product selection are discussed, and the robustness of results is checked whenever with 2007-2012 data whenever it is possible. In March 2015, the website was explored with a view to find the best way to extract price data. The only feasible solution appeared to successively crawl comparisons between Leclerc stores and each of their competitors. This implies that obtained data are less exhaustive than the full price database used by Qlmc to establish chain level comparisons.

\subsection{Leclerc competitors}

A total number of 575 Leclerc stores were found to be listed on the website. The comparison of each store with its respective selection of competitors yielded 2,390 pairs of stores, involving 1,815 unique Leclerc competitors (the number of pairs is larger since a store can be listed as a competitor of several Leclerc stores). Data were missing for 14 Leclerc stores and 51 Leclerc competitors. This implies that among competitors of the 561 Leclerc stores for which price data have been collected, 36 out of 1811 are missing ($\le 2 \%$).

Table~\ref{tab:qlmc_chain_repr} provides an overview of store coverage for the ten national chains compared on Qlmc. Coverage is high and rather close to coverage in the full qlmc sample for chains which are characterized by large store surfaces: Auchan, Carrefour, Cora, Geant and Leclerc. This can be explained by the fact that Leclerc is present across all regions and operates rather large stores. Regarding chains with smaller store formats, coverage is lower in percentage and also generally with respect to the full qlmc sample (e.g. 22\% for Casino vs. 39\% claimed by the website). Two natural explanations are the fact that stores from these chains are generally less likely to be considered relevant local competitors for Leclerc stores, and the slower development of "drive" within smaller stores (which make price collection less costly).

Table~\ref{tab:qlmc_comp} provides an overview of competition according to qlmc comparisons (the website does not claim to be comprehensive). On average, a Leclerc store is compared with 5 competitors, and over 50\% over Leclerc stores are compared with a store located within 2 km. The furthest competitor is generally within 30 km (15 km for almost half of them), except for 28 stores. For 14 Leclerc stores, the closest listed store if over 10 km away. No store meets these two criteria, hence it does not seem obvious that the lack or omission of nearby competitors led to include stores beyond reasonable distance. For instance, the Leclerc outlet which has the furthest competitor in the data (67 km) is listed with 7 competitors, of which 5 are located within 30 km.

\subsection{Products}

Price records obtained from the website include all products used in each store level comparison. As a consequence, there are 12,318 unique products in the data as of March 2015. Table~\ref{tab:qlmc_sections} provides an overview of the relative weights of each section in terms of product number and value. The five largest sections, regardless of the criterion, are Fresh products, Health and Beauty, Savoury Grocery, Sweet Grocery and Drinks. Families within each sections are detailed in table~\ref{tab:qlmc_families}. Drinks and Health and Beauty products tend to have a larger value than other categories', so that they account for a significantly higher share in terms of value than product count.

\subsection{Chain and store level comparisons}

Results for chain level comparisons performed according to the methodology described in the website documentation are reported in table~\ref{tab:qlmc_chain_comparisons}. Despite the fact that data collected differ from these used by Qlmc, results remain consistent, and are shown to be relatively robust to manipulations in product choice. Geant Casino is the second cheapest chain as of March 2015, only 1.5\% more expensive than Leclerc (1.8\% according to Qlmc). Dropping the 20\% products which weigh in most favorably for Leclerc reduces the comparison result to 0.4\%.

\subsection{Comparison history}

Collected data can be used to compare prices across periods and therefore to explain part of the changes recorded in successive chain price comparisons displayed by the website. Inflation between two successive periods is computed by comparing the sum of average prices of all products available over the two periods. Variations can then multiplied to obtain statistics over longer periods.

Leclerc prices between May 2007 and May 2012 have increased by 1.13\% (average annual increase of 0.25\%). Until May 2011, other chain display similarly low variations. This translates in a relative status quo in chain comparison results. Geant Casino is then the most expensive chain (+6\% to +10\%vs. Leclerc), followed by Cora and Carrefour Market (+5\%). Auchan, Carrefour, Geant Casino, Intermarche and Systeme U display rather similar price levels (+3-4\%).
After May 2011, most chains display a progressive loss in competitivess as compared to Leclerc. A remarkable exception is Geant Casino which, after a peak at +13.8\% vs. Leclerc in September 2012, becomes increasingly price competitive from May 2013 on. Carrefour appears to be willing to match this progression in 2014, and then to give up in 2015. As a consequence, while Geant Casino was the most expensive chain at the beginning of the period studied, it is the closest competitor of Leclerc in terms of price level as of March 2015 (+1.3\% vs. Leclerc, to be compared with +12.2\% in March 2013).
Intra-chains comparisons between May 2014 and March 2015 suggest that the relative loss in price competitiveness of Carrefour actually results from a mild change in prices by Carrefour (-1.4\%) constrasting with significant cuts implemented by other chains (e.g. -4.3\% for Auchan, -5.1\% for Leclerc, -5.2\% for Intermarche). Geant Casino achieves its unprecedented level of price competitiveness through an 8.5\% decrease.

\section{Measuring and accounting for price dispersion}

Since its creation in 2007, Qlmc prominently displays aggregate comparisons with its major national competitors. On the one hand, such information may be considered relevant by consumers willing to adopt a rule of thumb which weighs the cost/time of transportation with prices and products expected to be offered in a store of a given chain. On the other hand, without being deceptive per se, such comparisons could simply reflect differences in store characteristics such as location, size, market competitiveness etc. In order to address this issue, this section first investigates price dispersion within chains, and then discusses the possibility to account for prices through observed store and market characteristics.

\subsection{Static price dispersion}

French supermarket chains are known not to implement uniform national pricing policies. Empirical investigations however reveal different degrees of price dispersion within chains. Geant Casino, in particular, stands out in terms of price concentration. On average, a product is sold at the very same price in 91\% of its stores, and a store has 80\% of its products which follow a "standard" price (a product is considered to have a standard price whenever its most common price is set by 50\% or more of the stores). In terms of product price concentration, the closest followers are Systeme U and Leclerc (respective 39\% and 38\% on average). Leclerc distinguishes itself by having also a relatively strong concentration at the store level. At the median Leclerc store, 38\% products follow "standard" prices (17\% for Systeme U).

Of all comparisons between chains exhibiting similar price levels, the Leclerc vs. Geant Casino comparison is the most stable across stores, and within stores across products. Table~\ref{tab:static_compa_15km} shows that among 215 pairs, Geant Casino is +1.4\% more expensive on average, and Leclerc is less expensive in 85.1\% of the pairs.

Todo: add regression results and comments

\subsection{Dynamic price dispersion}

Descriptives statics of dynamic price dispersion are reported in Table~\ref{tab:dynamic_compa_15km}. Among 114 store comparisons involving a Leclerc and a Geant Casino, 4.4\% are won by a different store in the two periods. On average, 21.2\% of products taken into account in the comparison changed order between the two periods i.e were strictly cheaper at Leclerc in first period and became strictly cheaper at Geant Casino in second period or the reverse.

Todo: add regression results and comments

\newpage

\bibliography{references}

\newpage

\appendix

\section{Store data}

\begin{table}[H]
\begin{threeparttable}
\renewcommand{\arraystretch}{0.7}% Tighter
\caption{Representation of major national chains}\label{tab:qlmc_chain_repr}
\small
\begin{tabular}{lr|rr|rr}
\toprule
          & France & \multicolumn{2}{c|}{QLMC} & \multicolumn{2}{c}{Data} \\
          & Nb stores & Nb stores & Coverage & Nb stores & Coverage \\
\midrule
    Auchan & 142   & 125   & 88\%  & 112   & 79\% \\
    Carrefour & 222   & 188   & 85\%  & 171   & 77\% \\
    Carrefour Market & 925   & 421   & 46\%  & 239   & 26\% \\
    Casino & 392   & 151   & 39\%  & 76    & 19\% \\
    Cora  & 58    & 58    & 100\% & 54    & 93\% \\
    Geant & 108   & 108   & 100\% & 92    & 85\% \\
    Intermarche & 1,770 & 1,022 & 58\%  & 530   & 30\% \\
    Leclerc & 579   & 579   & 100\% & 561   & 97\% \\
    Simply Market & 305   & 50    & 16\%  & 49    & 16\% \\
    Systeme U & 1,030 & 632   & 61\%  & 413   & 40\% \\
\midrule		
    Total & 5,531 & 3,334 & 60\%  & 2,297 & 42\% \\
\bottomrule
\bottomrule
\end{tabular}
\begin{tablenotes}
      \small
      \item Data about store chains were provided by LSA, which is the source used by Qlmc.
\end{tablenotes}
\end{threeparttable}
\end{table}

\begin{table}[H]
\renewcommand{\arraystretch}{0.7}% Tighter
\caption{Overview of competition around the 575 Leclerc stores in Qlmc}\label{tab:qlmc_comp}
\small
\begin{tabular}{lr|rrrr}
\toprule
\toprule
{}         & Nb          &    \multicolumn{4}{c}{Distance (km) to} \\
{}         & competitors &    closest & furthest & mean & median \\
\midrule
Mean  &   5.0 &   2.4 &  15.9 &   8.8 &     8.5 \\
Std   &   1.6 &   2.5 &   9.7 &   5.1 &     6.0 \\
Min   &   1.0 &   0.1 &   0.9 &   0.8 &     0.5 \\
Q10   &   3.0 &   0.7 &   4.6 &   3.0 &     2.5 \\
Q25   &   4.0 &   1.1 &   8.4 &   4.8 &     3.7 \\
Q50   &   5.0 &   1.8 &  15.3 &   7.8 &     6.5 \\
Q75   &   6.0 &   2.7 &  21.5 &  12.3 &    12.5 \\
Q90   &   7.0 &   4.7 &  26.3 &  15.7 &    18.0 \\
Max   &  12.0 &  21.1 &  67.0 &  28.6 &    28.5 \\
\bottomrule
\bottomrule
\end{tabular}
\end{table}

\newpage

\section{Product data}

\begin{table}[H]
\renewcommand{\arraystretch}{0.7}% Tighter
\caption{Overview of product section weights}\label{tab:qlmc_sections}
\small
\begin{tabular}{lrr|rr|rr}
\toprule
\toprule
& \multicolumn{2}{c}{All products} & \multicolumn{2}{c|}{$\ge$ 500 obs} & \multicolumn{2}{c|}{$\ge$ 700 obs} \\
& Nb \% & Value \% & Nb \% & Value \% & Nb \% & Value \% \\
\midrule
    Baby and dietetic food & 4.7   & 4.3   & 3.9   & 3.0   & 3.3   & 2.4 \\
    Drinks & 10.0  & 15.3  & 10.9  & 20.4  & 11.1  & 21.9 \\
    Fresh products & 21.1  & 15.5  & 19.8  & 16.7  & 18.4  & 15.2 \\
    Frozen food & 3.0   & 3.1   & 3.0   & 3.9   & 2.4   & 3.1 \\
    Health and beauty & 17.3  & 26.9  & 11.5  & 12.8  & 12.4  & 13.4 \\
    Home and textile & 2.5   & 3.4   & 0.5   & 0.7   & 0.3   & 0.4 \\
    Household & 5.5   & 6.8   & 5.5   & 6.8   & 5.8   & 7.2 \\
    Pets  & 1.9   & 2.8   & 3.0   & 4.4   & 3.0   & 4.5 \\
    Savoury grocery & 16.5  & 9.4   & 19.6  & 12.5  & 20.4  & 12.6 \\
    Sweet grocery & 17.0  & 12.3  & 22.1  & 18.8  & 22.8  & 19.2 \\
    Vegetables and fruits & 0.5   & 0.4   & 0.2   & 0.2   & 0.2   & 0.2 \\
\midrule
    Total & 100.0 & 100.0 & 100.0 & 100.0 & 100.0 & 100.0 \\
    Total Nb or Value (euros) & 12,318 & 43,883 & 3,467 & 9,138 & 2,578 & 6,682 \\
\bottomrule
\bottomrule
\end{tabular}
\end{table}

\begin{table}[H]
\renewcommand{\arraystretch}{0.8}% Tighter
\caption{Overview of product families within sections}\label{tab:qlmc_families}
\small
\begin{tabularx}{\linewidth}{l >{\setlength{\baselineskip}{0.75\baselineskip}}X}
%\begin{tabular}{p{0.3\linewidth}p{0.7\linewidth}}
\toprule
\toprule
    Section & Families \\
    \midrule
    Baby and dietetic food (573) & Baby food (418); Dietetic products (155) \\
    Drinks (1,233) & Beer and Spirits (443); Fizzy drinks and Cola (244); Water (176); Juices and Smoothies (110); Squash and Cordial (101); Wine, Champagne and Cider (159) \\
    Fresh products (2,595) & Butter and Cream (199); Meat (490); Cheese (491); Milk and eggs (150); Fish (98); Delicatessen (660); Yoghurts and Chilled Desserts (507) \\
    Frozen food (368) & Ice cream and Frozen yoghurt (101); Frozen vegetables and fries (91); Frozen pizzas, pies and ready meals (128); Frozen Meat and Fish (48) \\
    Health and Beauty (2,127) & Kitchen Roll and Tissues (86); Oral care (169); Feminine care and Baby changing (138); Drugstore (97); Haircare (558); Face and body skincare (951); Men toiletries (128) \\
    Home and textile (308) & DIY and Car (9); Kitchen and dining room (50); Home Office (171); Batteries, lightbulbs and plugs (54) \\
    Household (679) & Air fresheners and insect killers (118); Laundry (124); Cloths, Gloves and Scourers (45); Cleaning (225); Dishwashing (64); Specialist laundry and Washing machine cleaner (103) \\
    Pets (239) & Cat and dog food (233); Litter (6) \\
    Savoury grocery (2,032) & Snacks (214); Condiments and Spices (609); Canned goods (406); Precooked dishes (205); Pasta, Rice and Flour (328); Soups (270) \\
    Sweet grocery (2,099) & Biscuits (294); Coffee and Tea (368); Chocolates ans sweets (450); Desserts, Sugar and Sweeteners (318); Breakfast (453); Cakes (215) \\
    Vegetables and fruits (65) & Fruits (65) \\
\bottomrule
\bottomrule
\end{tabularx}
%\end{tabular}
\end{table}

\section{Comparison data}

\begin{table}[H]
\begin{threeparttable}
\renewcommand{\arraystretch}{0.7}% Tighter
\caption{Comparisons at the chain level}\label{tab:qlmc_chain_comparisons}
\small
\begin{tabular}{l|rr|rr|rrrr}
\toprule
\toprule
          & \multicolumn{2}{c|}{Nb stores} &  \multicolumn{2}{c|}{Nb products} & \multicolumn{4}{c}{Comparison vs. Leclerc} \\
           & Qlmc  & Data  & Qlmc  & Data  & Qlmc  & Data  & Bias 10\% & Bias 20\% \\
\midrule
    Auchan & 125   & 112   & 1,976 & 2,382 & +7.6\% & +6.5\% & +5.5\% & +5.0\% \\
    Carrefour & 188   & 171   & 1,294 & 1,284 & +7.8\% & +8.2\% & +7.0\% & +6.0\% \\
    Carrefour market & 421   & 239   & 2,032 & 3,401 & +13.5\% & +12.4\% & +11.6\% & +10.2\% \\
    Casino & 151   & 76    & na    & 1,650 & +16.7\% & +16.8\% & +15.8\% & +15.4\% \\
    Cora  & 58    & 54    & 1,326 & 2,994 & +10.2\% & +9.4\% & +8.3\% & +7.3\% \\
    Geant Casino & 108   & 92    & 1,582 & 1,582 & +1.8\% & +1.5\% & +0.7\% & +0.4\% \\
    Intermarche & 1,022 & 530   & 1,971 & 6,287 & +7.0\% & +7.1\% & +5.8\% & +5.0\% \\
    Simply market & 50    & 49    & na    & 1,070 & +12.9\% & +13.4\% & +11.6\% & +11.2\% \\
    Systeme U & 632   & 413   & 2,386 & 4,565 & +6.7\% & +5.8\% & +4.8\% & +4.7\% \\
\bottomrule
\bottomrule
\end{tabular}
\begin{tablenotes}
      \small
      \item Comparisons are based on 561 Leclerc stores (vs. 581 in Qlmc). In the column "Bias 10\%", the 10\% products which compare most favorably for Leclerc in terms of percent price difference are dropped.
\end{tablenotes}
\end{threeparttable}
\end{table}

\begin{table}[H]
\begin{threeparttable}
\renewcommand{\arraystretch}{0.7}% Tighter
\caption{Comparisons between Leclerc stores and their competitors by chain}\label{tab:qlmc_store_comparisons}
\small
\begin{tabular}{lr|rrrrrrr}
\toprule
\toprule
          & Nb    & \multicolumn{7}{c}{Comparison of Leclerc stores vs. competitors by chain} \\
          & pairs & Mean  & Std   & Min   & Q25  & Q50 & Q75  & Max \\
\midrule
    Auchan & 118   & +6.5\% & 3.3\% & +1.6\% & +4.1\% & +5.7\% & +8.3\% & +19.5\% \\
    Carrefour & 175   & +8.2\% & 5.2\% & -3.5\% & +5.8\% & +8.1\% & +9.4\% & +36.2\% \\
    Carrefour market & 235   & +13.8\% & 3.3\% & +1.3\% & +11.7\% & +13.5\% & +15.8\% & +24.5\% \\
    Casino & 57    & +17.9\% & 4.8\% & +0.5\% & +16.8\% & +18.7\% & +21.0\% & +27.5\% \\
    Cora  & 57    & +8.6\% & 2.4\% & +3.6\% & +6.7\% & +8.4\% & +10.3\% & +15.6\% \\
    Geant Casino & 99    & +1.8\% & 1.5\% & -0.6\% & +0.7\% & +1.3\% & +2.3\% & +5.3\% \\
    Intermarche & 525   & +7.1\% & 2.8\% & +2.0\% & +5.4\% & +6.6\% & +8.2\% & +28.4\% \\
    Simply market & 49    & +13.4\% & 6.2\% & +6.5\% & +9.8\% & +10.6\% & +15.4\% & +31.8\% \\
    Systeme U & 355   & +6.7\% & 4.0\% & +1.1\% & +3.8\% & +5.8\% & +8.7\% & +26.0\% \\
\bottomrule
\bottomrule
\end{tabular}
\begin{tablenotes}
      \small
      \item Pairs are required to include 400 products or more. There are 118 comparisons between a Leclerc store and an Auchan store. On average, an Auchan store is 6.5\% more expensive than its Leclerc competitor.
\end{tablenotes}
\end{threeparttable}
\end{table}

\newpage

\section{Chain pricing policies}

\begin{table}[H]
\begin{threeparttable}
\renewcommand{\arraystretch}{0.7}% Tighter
\caption{Distribution of the frequencies of the most common price per product}\label{tab:qlmc_prod_freq}
\small
\begin{tabular}{lrrrrrrrr}
\toprule
\toprule
{}                 &  Nb &  Mean &  Std &  Min &  Q25 &  Q50 &  Q75 &  Max \\
\midrule
Auchan             &   416 &  0.19 & 0.11 & 0.05 & 0.12 & 0.16 & 0.22 & 0.63 \\
Carrefour          &   319 &  0.29 & 0.17 & 0.07 & 0.17 & 0.23 & 0.36 & 0.87 \\
Carrefour Market   &   777 &  0.33 & 0.19 & 0.11 & 0.20 & 0.26 & 0.42 & 1.00 \\
Geant Casino       &   417 &  0.89 & 0.10 & 0.45 & 0.83 & 0.91 & 0.97 & 1.00 \\
Casino             &   157 &  0.37 & 0.15 & 0.06 & 0.29 & 0.33 & 0.44 & 0.86 \\
Cora               &   364 &  0.20 & 0.11 & 0.06 & 0.14 & 0.17 & 0.23 & 0.90 \\
Intermarche        & 1,326 &  0.25 & 0.19 & 0.05 & 0.13 & 0.18 & 0.29 & 0.97 \\
Leclerc            & 1,788 &  0.38 & 0.23 & 0.03 & 0.14 & 0.38 & 0.59 & 0.95 \\
Systeme U          & 1,077 &  0.39 & 0.12 & 0.09 & 0.32 & 0.37 & 0.44 & 0.91 \\
\bottomrule
\bottomrule
%\multicolumn{9}{p{0.8\textwidth}}{\footnotesize Reading note: On average, 38\% of all Leclerc stores set the very same price for a given product.}
\end{tabular}
\begin{tablenotes}
      \small
      \item On average, 38\% of all Leclerc stores set the very same price for a given product.
\end{tablenotes}
\end{threeparttable}
\end{table}

\begin{table}[H]
\begin{threeparttable}
\renewcommand{\arraystretch}{0.7}% Tighter
\caption{Distribution of the frequencies of "standard" prices per store}\label{tab:qlmc_store_freq_50}
\small
\begin{tabular}{lrrrrrrrr}
\toprule
\toprule
{}                &  Nb &  Mean &  Std &  Min &  Q25 &  Q50 &  Q75 &  Max \\
\midrule
Auchan            & 107 &  0.01 & 0.01 & 0.00 & 0.00 & 0.01 & 0.01 & 0.02 \\
Carrefour         & 146 &  0.07 & 0.03 & 0.00 & 0.05 & 0.08 & 0.09 & 0.13 \\
Carrefour Market  & 223 &  0.12 & 0.05 & 0.00 & 0.10 & 0.13 & 0.15 & 0.21 \\
Geant Casino      &  91 &  0.80 & 0.23 & 0.06 & 0.70 & 0.94 & 0.96 & 0.98 \\
Casino            &  74 &  0.04 & 0.02 & 0.01 & 0.03 & 0.05 & 0.06 & 0.09 \\
Cora              &  54 &  0.01 & 0.00 & 0.00 & 0.01 & 0.01 & 0.02 & 0.02 \\
Intermarche       & 513 &  0.07 & 0.03 & 0.00 & 0.05 & 0.08 & 0.10 & 0.17 \\
Leclerc           & 552 &  0.28 & 0.12 & 0.01 & 0.19 & 0.31 & 0.38 & 0.51 \\
Systeme U         & 409 &  0.08 & 0.05 & 0.00 & 0.04 & 0.06 & 0.12 & 0.17 \\
\bottomrule
\bottomrule
\end{tabular}
\begin{tablenotes}
      \small
      \item A price is considered "standard" if it is shared by 50\% of the chain stores or more. On average, a Leclerc store sells 28\% of its products at a "standard" price.
\end{tablenotes}
\end{threeparttable}
\end{table}

\section{Rank reversals}

\begin{table}[H]
\begin{threeparttable}
\renewcommand{\arraystretch}{0.7}% Tighter
\caption{Static store level comparisons (15 km - 100 obs min)
}\label{tab:static_compa_15km}
\small
\begin{tabular}{llrrrrrrr}
\toprule
\toprule
    \textbf{} &       & Nb    & B vs A avg & Pairs won & \multicolumn{4}{c}{Share of products (avg \%)} \\
    Chain A & Chain B & pairs & comparison & by A (\%) & A wins & B wins & Draw & Reversed \\
\midrule
    Leclerc & Geant Casino & 215   & +1.4\% & 85.1  & 61.8  & 22.4  & 15.8  & 20.4 \\
    Leclerc & Carrefour & 555   & +9.1\% & 98.4  & 78.5  & 15.1  & 6.4   & 14.7 \\
    Geant Casino & Carrefour & 89    & +7.6\% & 98.9  & 70.8  & 25.1  & 4.1   & 25.1 \\
    Carrefour & Auchan & 191   & -0.3\% & 51.8  & 46.3  & 44.3  & 9.4   & 28.9 \\
    Carrefour & Intermarche & 365   & -1.0\% & 38.6  & 45.8  & 51.2  & 3.0   & 34.0 \\
    Carrefour & Systeme U & 196   & +2.6\% & 60.7  & 57.1  & 38.8  & 4.1   & 27.3 \\
    Auchan & Intermarche & 212   & +0.8\% & 61.8  & 54.0  & 43.0  & 3.0   & 32.9 \\
    Auchan & Systeme U & 145   & +3.1\% & 66.2  & 60.5  & 35.2  & 4.3   & 27.0 \\
    Intermarche & Systeme U & 490   & +1.0\% & 51.2  & 51.5  & 41.3  & 7.3   & 25.3 \\
\bottomrule
\bottomrule
\end{tabular}
\begin{tablenotes}
      \small
      \item Among 215 pairs of Leclerc and Geant Casino competitors, Geant Casino is +1.4\% more expensive on average, and Leclerc is less expensive in 85.1\% of the pairs. On average, a Leclerc sells 61.8\% of products strictly cheaper than its Geant Casino competitor. Regardless of whether Leclerc or Geant Casino wins the overall comparison, on average, the loser i.e. most expensive store is strictly cheaper on 20.4\% of products.
\end{tablenotes}
\end{threeparttable}
\end{table}

\begin{table}[H]
\begin{threeparttable}
\renewcommand{\arraystretch}{0.7}% Tighter
\caption{Dynamic store level comparisons (15 km - 100 obs min)
}\label{tab:dynamic_compa_15km}
\small
\begin{tabular}{llrrr}
\toprule
\toprule
    \textbf{} &       &       & \multicolumn{2}{c}{Dynamic "Rank reversals"} \\
    Chain A & Chain B & Nb pairs & Pairs (\%) & Products (\%) \\
\midrule
    Leclerc & Geant Casino & 114   & 4.4   & 21.2 \\
    Leclerc & Carrefour & 152   & 5.9   & 24.6 \\
    Geant Casino & Carrefour & 46    & 71.7  & 42.5 \\
    Carrefour & Auchan & 49    & 42.9  & 38.0 \\
    Carrefour & Intermarche & 119   & 53.8  & 38.6 \\
    Carrefour & Systeme U & 102   & 48.0  & 37.2 \\
    Auchan & Intermarche & 86    & 22.1  & 32.4 \\
    Auchan & Systeme U & 101   & 34.7  & 29.9 \\
    Intermarche & Systeme U & 322   & 32.8  & 30.5 \\
\bottomrule
\bottomrule
\end{tabular}
\begin{tablenotes}
      \small
      \item Among 114 store comparisons involving a Leclerc and a Geant Casino, 4.4\% are won by a different store in the two periods (draws can be neglected as they virtually never happen). On average, 21.2\% of products taken into account in the comparison changed order between the two periods i.e were strictly cheaper at Leclerc in first period and became strictly cheaper at Geant Casino in second period or the reverse.
\end{tablenotes}
\end{threeparttable}
\end{table}

\end{document}